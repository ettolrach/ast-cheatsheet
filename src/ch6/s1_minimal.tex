\wde{6.1.1 Equivalence relation} An \textit{equivalence relation} on a set $X$ is a reflexive, transitive, symmetric relation on $X$.
\wl{6.1.4} The intersection of any set of equiv. rels. is an equiv. rel.
\wde{6.1.5 Generated equiv. rel.} Let $R$ be a relation on a set $X$. The \textit{equiv. rel. generated by $R$} is the intersection of all equiv. rels. on $X$ containing $R$. That is, the equiv. rel. generated by $R$ is $\bigcap\{ E \in \Pow(X \times X) \colon E \text{ is an equiv. rel. on } X \text{ and } R \subseteq E \}$.
\wl{6.1.6} Let $R$ be a relation on a set $X$, and let $\sim$ be generated by $R$. Then: \textbf{(i)} $\sim$ is an equiv. rel. containing $R$; \textbf{(ii)} any equiv. rel. containing $R$ also contains $\sim$.
