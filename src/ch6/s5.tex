\wde{6.5.4 $I$-indexed family of sets} Let $I$ be aa set. An \textit{$I$-indexed family of sets} is a set $X$ with a function $p \colon X \to I$. We call $I$ the \textit{indexing set}, the fibre $p^{-1}(i)$ as $X_i$, and call $X_i$ the $i$th \textit{member} of the family.
\wl{6.5.6} Let $Y$ and $I$ be sets, and let $(A_i \hookrightarrow Y)_{i \in I}$ be an $I$-indexed family of subsets of $Y$. Then there exists an $I$-indexed family of sets $X \to I$ such that $X_i \cong A_i$ for all $i \in I$.
\wde{6.5.8 Family isomorphism} Let $I$ be a set. Two $I$-indexed families $X \xrightarrow{p} I$ aand $X' \xrightarrow{p'} I$ are said to be \textit{isomorphic} if they are isomorphic over $I$ (definition 3.2.12). We often write $X \xrightarrow{p} I$ as $(X_i)_{i \in I}$ and call $p$ the \textit{projection} of the family.
\wpr{6.5.12} Let $(X_i)_{i \in I}$ be a family of sets. Then there exists a set not isomorphic to aanay of the sets $X_i (i \in I)$.
