\wde{6.1.1 Equivalence relation} An \textit{equivalence relation} on a set $X$ is a reflexive, transitive, symmetric relation on $X$.
\wl{6.1.4} The intersection of any set of equivalence relations is an equivalence relation. That is, let $X$ be a set, let $\mathscr{E} \subseteq \Pow(X \times X)$, and suppose that each element $E \in \mathscr{E}$ is an equivalence relation on X. Then $\bigcap\mathscr{E}$ is an equivalence relation on $X$.
\wde{6.1.5 Generated equivalence relation} Let $R$ be a relation on a set $X$. The \textit{equivalence relation generated by $R$} is the intersection of all equivalence relations on $X$ containing $R$. That is, the equivalence relation generated by $R$ is $\bigcap\{ E \in \Pow(X \times X) \colon E \text{ is an equiv. rel. on } X \text{ and } R \subseteq E \}$.
\wl{6.1.6} Let $R$ be a relation on a set $X$, and let $\sim$ be the equivalence relation on $X$ generated by $R$. Then: \textbf{(i)} $\sim$ is an equivalence relation containing $R$; \textbf{(ii)} any equivalence relation containing $R$ also contains $\sim$. I.e. $\sim$ is the smallest equivalence relation containing $R$.
