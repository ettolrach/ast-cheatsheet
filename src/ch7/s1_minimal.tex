\wt{7.1.1 Principle of induction} Let $A \subseteq \N$. Suppose that $0 \in_{\N} A$ and that for all $n \in \N$, we have $n \in_{\N} A \implies s(n) \in_{\N} A$. Then $A = \N$.
\wl{7.1.3} The function $\binom{0}{s} \colon \mathbf{1} + \N \to \N$ is an iso.
\wde{7.1.12} For $n \in \N$, we write $B(n) = \{ i \in \N \colon i < n \}$.
\wl{7.1.13} $B(s(n)) \cong B(n) + \mathbf{1}$ for all $n \in \N$.
\wpr{7.1.14} For all $m, n \in \N$, \textbf{(i)} $B(m + n) \cong B(m) + B(n)$, and $B(0) \cong \emptyset$; \textbf{(ii)} $B(m \cdot n) \cong B(m) \times B(n)$, and $B(1) \cong \mathbf{1}$; \textbf{(iii)} $B(m^n) \cong B(m)^{B(n)}$.
\wl{7.1.15} Let $X$ and $Y$ be sets. Then: \textbf{(i)} $X \cong Y \iff X + \mathbf{1} \cong Y + \mathbf{1}$; \textbf{(ii)} $X \leq Y \iff X + \mathbf{1} \leq Y + \mathbf{1}$.
\wl{7.1.16, 7.1.17} $m \leq n \iff B(m) \leq B(n)$ and $m = n \iff B(m) \cong B(n)$, for $m, n \in \N$.
\wl{7.1.26} The relation $\leq$ on $\N$ is total.
