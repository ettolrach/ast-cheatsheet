\wt{7.1.1 Principle of induction} Let $A \subseteq \N$. Suppose that $0 \in_{\N} A$ and that for all $n \in \N$, we have $n \in_{\N} A \implies s(n) \in_{\N} A$. Then $A = \N$.
\wl{7.1.3} The function $\binom{0}{s} \colon \mathbf{1} + \N \to \N$ is an isomorphism.
\wpr{7.1.4} \textbf{(i)} Every natural number is $0$ or a successor. \textbf{(ii)} $0$ is not a successor. \textbf{(iii)} If $s(n) = s(m)$ then $n = m$, for $n, m \in N$.
\wl{7.1.5} There exists unique function $+ \colon \N \times \N \to \N$ such that $m + 0 = m$ and $m + s(n) = s (m + n)$ for all $m, n \in \N$.
\wl{7.1.6} There exists a unique function $\cdot \colon \N \times \N \to \N$ such that $m \cdot 0 = 0$ and $m \cdot s(n) = m \cdot n + m$ for all $m, n \in \N$.
\wde{no number} We call $+$ \textit{addition} and $\cdot$ \textit{multiplication} on $\N$ and we define $1$ as $s(0)$.
\wl{7.1.7} There exists a unique function $E \colon \N \times \N \to \N$ such that, writing $E(m, n)$ as $m^n$, $m^0 = 1$ and $m^{s(n)} = m^n \cdot m$ for all $m, n \in \N$.
\wl{7.1.8} $s(n) = n + 1$ for all $n \in \N$.
\wde{7.1.9 $\leq$ on $\N$} The relation $\leq$ on $\N$ is defined by $m \leq n \iff k + m = n$ for some $k \in N$. We define $<$, $\geq$, and $>$ as usual.
\wl{7.1.10} Let $m, n \in \N$. \textbf{(i)} $m \leq n \iff s(m) \leq s(n)$. \textbf{(ii)} $m < n \iff s(m) < s(n)$. \textbf{(iii)} $n \geq 0$ \textbf{(iv)} $n \leq 0 \implies n = 0$ \textbf{(v)} $m < s(n) \iff m \leq n$.
\wde{7.1.12} For $n \in \N$, we write $B(n) = \{ i \in \N \colon i < n \}$.
\wl{7.1.13} $B(s(n)) \cong B(n) + \mathbf{1}$ for all $n \in \N$.
\wpr{7.1.14} For all $m, n \in \N$, \textbf{(i)} $B(m + n) \cong B(m) + B(n)$, and $B(0) \cong \emptyset$; \textbf{(ii)} $B(m \cdot n) \cong B(m) \times B(n)$, and $B(1) \cong \mathbf{1}$; \textbf{(iii)} $B(m^n) \cong B(m)^{B(n)}$.
\wl{7.1.15} Let $X$ and $Y$ be sets. Then: \textbf{(i)} $X \cong Y \iff X + \mathbf{1} \cong Y + \mathbf{1}$; \textbf{(ii)} $X \leq Y \iff X + \mathbf{1} \leq Y \mathbf{1}$.
\wl{7.1.16} $m \leq n \iff B(m) \leq B(n)$, for $m, n \in \N$.
\wl{7.1.17} $m = n \iff B(m) \cong B(n)$, for $m, n \in \N$.
\wpr{7.1.18} On $\N$, addition and multiplication are commutative, associative, have identity ($0$ and $1$ respectively). Multiplication distributes over addition. We have standard power laws too. Importantly, $p^0 = 1$ for all $p \in \N$.
\wpr{7.1.19} \textbf{(i)} The relation $\leq$ on $\N$ is an order. \textbf{(ii)} Let $m, n \in \N$ with $m \leq n$. Then for all $p \in \N$ we have $m + p \leq n + p$, $mp \leq np$, and $m^p \leq n^p$; if also $p > 0$ then $p^m \leq p^n$.
\wl{7.1.21} Let $m, n \in \N$. If $m + n = 0$ then $m = n = 0$, and if $mn = 0$ then $m = 0$ or $n = 0$.
\wl{7.1.22} Let $m, n, p \in \N$. If $m + p = n + p$ then $m = n$.
\wl{7.1.24} Let $m, n, p \in \N$ with $p \neq 0$. If $m p = n p$ then $m = n$.
\wl{7.1.26} The relation $\leq$ on $\N$ is total.
