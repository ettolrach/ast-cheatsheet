\wde{n.n., Order isomorphism} An \textit{isomorphism of ordered sets} $X \to Y$ is an order-preserving bijection whose inverse is also order-preserving. We say that $X$ and $Y$ are \textit{(order) isomorphic}, and write $X \cong Y$.
\wde{8.1.3 Well order} $\leq$ on $W$ is a \textit{well order} if every nonempty subset of $W$ has a least element.
\wl{8.1.4} Every w.o. set is totally ordered.
\wde{8.1.7 Ordinal sum and product} Let $X = (X, \leq_1)$ and $Y = (Y, \leq_2)$ be ordered sets. \textbf{(i)} The \textit{ordinal sum} is $X + Y$ with the order relation $\leq$: for $z, z' \in X + Y$, we define $z \leq z'$ iff $(z, z' \in X \text{ and } z \leq_1 z')$ or $(z, z' \in Y \text{ and } z \leq_2 z')$ or $(z \in X \text{ and } z' \in Y)$. \textbf{(i)} The \textit{ordinal product} $X \times Y$ with this order relation $\leq$: for $(x, y), (x', y') \in X \times Y$, we define $(x, y) \leq (x', y')$ iff $(y <_2 y')$ or $(y = y' \text{ and } x \leq_1 x')$.
\wde{n.n.} A w.o. set is called a \textit{successor} if it's iso. to $V + \mathbf{1}$ for some w.o. set $V$, and a \textit{limit} otherwise.
\wl{8.1.13} A w.o. set is a successor iff it has a greatest element.
\wde{8.1.15 Downset} A subset $A$ of an ordered set $X$ is \textit{downwards closed}, or a \textit{downset}, if for all $x, y \in X$, $(x \leq y \text{ and } y \in A) \implies x \in A$.
\wl{8.1.17} Let $W$ be a w.o. set and let $A$ be a downset in $W$. Then $A = W$ or $\exists! w \in W$ such that $A = \ddownarrow w$.
\wt{8.1.18 Transfinite induction} Let $W$ be a w.o. set. Let $A$ be a subset of $W$ such that for all $w \in W$, $\ddownarrow w \subseteq A \implies w \in A$. Then $A = W$.
