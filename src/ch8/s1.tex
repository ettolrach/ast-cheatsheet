\wde{no number, Order isomorphism} An \textit{isomorphism of ordered sets} or \textit{order isomorphism} $X \to Y$ is an order-preserving bijection whose inverse is also order-preserving. We say that $X$ and $Y$ are \textit{order isomorphic}, or just \textit{isomorphic} and write $X \cong Y$.
\wde{8.1.3 Well order} An order $\leq$ on a set $W$ is a \textit{well order} if every nonempty subset of $W$ has a leaast element. A set together with a well order is called a \textit{well ordered set}.
\wl{8.1.4} Every well ordered set is totally ordered.
\wde{8.1.7 Ordinal sum and product} Let $X = (X, \leq_1)$ and $Y = (Y, \leq_2)$ be ordered sets. \textbf{(i)} The \textit{ordinal sum} of $X$ and $Y$ is the coproduct $X + Y$ with the following order relation $\leq$: for $z, z' \in X + Y$, we define $z \leq z'$ if and only if $(z, z' \in X \text{ and } z \leq_1 z')$ or $(z, z' \in Y \text{ and } z \leq_2 z')$ or $(z \in X \text{ and } z' \in Y)$. \textbf{(i)} The \textit{ordinal product} of $X$ and $Y$ is the product $X \times Y$ with the following order relation $\leq$: for $(x, y), (x', y') \in X \times Y$, we define $(x, y) \leq (x', y')$ if and only if $(y <_2 y')$ or $(y = y' \text{ and } x \leq_1 x')$. We call the ordinal product ordering the \textit{reverse lexicographic order}.
\wl{8.1.8} The ordinal sum and ordinal product of two well ordered sets are well ordered.
\wde{no number, Successor and limit} A well ordered set is called a \textit{successor} if it is isomorphic to $V + \mathbf{1}$ for some well ordered set $V$, and a \textit{limit} otherwise.
\wl{8.1.13} A well ordered set is a successor if and only if it has a greatest element.
\wde{8.1.15 Downset} A subset $A$ of an ordered set $X$ is \textit{downwards closed}, or a \textit{downset}, if for all $x, y \in X$, $(x \leq y \text{ and } y \in A) \implies x \in A$.
\wl{8.1.17} Let $W$ be a well ordered set and let $A$ be a downset in $W$. Then $A = W$ or there exists a unique $w \in W$ such that $A = \ddownarrow w$.
\wt{8.1.18 Transfinite induction} Let $W$ be a well ordered set. Let $A$ be a subset of $W$ such that for all $w \in W$, $\ddownarrow w \subseteq A \implies w \in A$. Then $A = W$.
