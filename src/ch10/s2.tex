\wde{10.2.1, (Un)countable} A set $X$ is \textit{countable} if $X \leq \N$, and \textit{uncountable} otherwise.
\wl{10.2.3} A set $X$ is countable iff $X$ is empty or there exists a surjection $\N \to X$.
\wc{10.2.4} Let $X$ and $Y$ be sets, with $X$ countable. If there exists an injection $Y \to X$ or a surjection $X \to Y$, then $Y$ is also countable.
\wpr{10.2.5} $\N \times \N \cong \N$.
\wl{10.2.7} \textbf{(i)} $\forall X, X', Y, Y'$ sets, if $X \leq X'$ and $Y \leq Y'$ then $X + Y \leq X' + Y'$, and $X \times Y \leq X' \times Y'$. \textbf{(ii)} $\forall X, X', Y$ sets, if $X \leq X'$ then $X^Y \leq X'^Y$, and $Y^X \leq Y^{X'}$, the latter under the restriction that $Y$ is nonempty.
\wc{10.2.8} Let $n \geq 1$. Let $X_1, \dots, X_n$ be countable sets. Then $X_1 \times \dots \times X_n$ is countable.
\wde{n.n.} Given a set $X$, we define $X^n$ for each $n \in \N$ recursively: $X^0 = \mathbf{1}$, and $X^{n + 1} = X^n \times X$ ($n \in \N$). Thus, $X^1 \cong X$, $X^2 \cong X \times X$, and so on.
\wc{10.2.9} If $X$ is countable, then $X^n$ is countable $\forall n \geq 1$.
\wl{10.2.11} Let $X$ be a set, and let $(A_i)_{i \in I}$ be a family of subsets of $X$. If $I$ is countable and $A_i$ is countable for each $i \in I$, then so too is $\bigcup_{i \in I} A_i$.
\wl{10.2.14} $\R$ is iso. to any real interval with more than one element.
\wpr{10.2.15} $R \cong \mathbf{2}^{\N}$. In particular, $\R$ is uncountable.
\wc{10.2.17} A real interval with more than one element is uncountable.
