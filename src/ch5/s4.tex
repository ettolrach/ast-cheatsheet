\wde{5.4.2 Property specifying a subset} Let $X$ be a set and let $P(x)$ be a property of elements $x \in X$. We say that $P(x)$ \textit{specifies a subset of $X$} if there exists $A \subseteq X$ such that for $x \in X$, $x \in_X A \iff P(x)$. Then we define $\{x \in X \colon P(x) \}$ to be this subset $A$ of $X$. We say that a subset \textit{exists} if $P(x)$ specifies a subset of $X$.
\wl{5.4.4} For sets and functions $X \rightrightarrows^f_g Y$, the subset $\{x \in X \colon f(x) = g(x) \}$ of $X$ exists. It is called the \textit{equalizer} of $f$ and $g$.
\wl{5.4.7} Let $X$ be a set and let $P(x)$ and $Q(x)$ be properties of elements $x \in X$. If $P(x)$ and $Q(x)$ specify subsets of $X$, then so do the following properties of elements $x \in X$: \textbf{(i)} not P(x); \textbf{(ii)} $P(x)$ and $Q(x)$; \textbf{(iii)} $P(x)$ or $Q(x)$; \textbf{(iv)} $P(x) \implies Q(x)$; \textbf{(v)} $P(x) \iff Q(x)$.
\wl{5.4.10} Let $f \colon X \to Y$ be a function. Let $Q(y)$ be a property of element $y \in Y$, and define a property $P(x)$ of elements $x \in X$ by $P(x) = Q(f(x))$. If $Q(y)$ specifies a subset of $Y$ then $P(x)$ specifies a subset of $X$.
\wl{5.4.12} Let $P(x, y)$ be a property of elements $(x, y) \in X \times Y$ that specifies a subset of $X \times Y$. Then $(\forall x \in X)P(x, y)$ and $(\exists x \in X)P(x, y)$ are properties of elements $y \in Y$ that specify subset of $Y$.
