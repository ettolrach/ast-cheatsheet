\wde{5.1.2 Relations between sets} Let $X$ and $Y$ be sets. A \textit{relation between $X$ and $Y$} is a subset of $X \times Y$. When $X = Y$, we call it a \textit{relation on $X$}.
\wde{n.n., Opposite relation} Let $X$ and $Y$ be sets and $R \subseteq X \times Y$ a relation. Then $R^{\op}$, the \textit{opposite of $R$}, is a relation from $Y$ to $X$, and $y R^{\op}x \iff xRy$, where $y \in Y$ and $x \in X$.
\wde{5.1.9, Symmetric relation} A relation $R$ on a set $X$ is \textit{symmetric} if $R^{\op} = R$, i.e. if $xRy \iff yRx$ for all $x, y \in X$.
\wl{5.1.11} Let $R$ be a relation between sets $X$ and $Y$. Then for each $x \in X$, there is a unique subset $R[x]$ of $y$ such that for $y \in Y$, $y \in_Y R[x] \iff xRy$.
\wl{5.1.14} Let $X$, $Y$, and $Z$ be sets, let $R$ be a relation between $X$ and $Y$, and let $S$ be a relation between $Y$ and $Z$. Then there is a unique relation $S \circ R$ between $X$ and $Z$ such that for all $x \in X$ and $z \in Z$, $x(S \circ R)z \iff (xRy \text{ and } ySz \text{ for some } y \in Y)$. We call the relation $S \circ R$ the composite of $R$ and $S$.
\wl{5.1.17} \textbf{(i)} Composite relations are associative. \textbf{(ii)} Let $R$ be a relation between $X$ and $Y$. Then $R \circ \triangle_X = R = \triangle_Y \circ R$. \textbf{(iii)} Let $R$ be a relation between $X$ and $Y$ and $S$ a relation between $Y$ and $Z$. Then $(S \circ R)^{\op} = R^{\op} \circ S^{\op}$.
\wde{5.1.18} A relation $R$ on a set $X$ is: \textbf{(i)} \textit{reflexive} if $xRx$ for all $x \in X$; \textbf{(ii)} \textit{transitive} if $(xRy \text{ and } yRz) \implies xRz$, for $x, y, z \in X$; \textbf{(ii)} \textit{antisymmetric} if $(xRy \text{ and } yRx) \implies x = y$, for $x, y \in X$.
\wde{5.1.21 Order} An \textit{order} (or \textit{order relation} or \textit{ordering}) on a set $X$ is a relation on $X$ that is reflexive, transitive, and antisymmetric. A set together with an order on it is called an \textit{ordered set}.
\wde{5.1.23 Total order} An order $\leq$ on a set $X$ is \textit{total} if for all $x, y \in X$, we have $x \leq y$ or $y \leq x$. Then $X$ together with $\leq$ is called a \textit{totally ordered set}.
