\wde{5.1.2 Relations between sets} Let $X$ and $Y$ be sets. A \textit{relation between $X$ and $Y$} is a subset of $X \times Y$. When $X = Y$, we call it a \textit{relation on $X$}. We also have the \textit{opposite of $R$}, a relation from $Y$ to $X$, and $y R^{\op}x \iff xRy$, where $y \in Y$ and $x \in X$. if $R^{\op} = R$, i.e. if $xRy \iff yRx$ for all $x, y \in X$, it is \textit{symmetric}.
\wl{5.1.11} Let $R$ be a relation between sets $X$ and $Y$. Then for each $x \in X$, there is a unique subset $R[x]$ of $y$ such that for $y \in Y$, $y \in_Y R[x] \iff xRy$.
\wl{5.1.14} The \textit{composite} of relations: for all $x \in X$ and $z \in Z$, $x(S \circ R)z \iff (xRy \text{ and } ySz \text{ for some } y \in Y)$.
\wl{5.1.17} \textbf{(i)} Composite relations are associative. \textbf{(ii)} Let $R$ be a relation between $X$ and $Y$. Then $R \circ \triangle_X = R = \triangle_Y \circ R$. \textbf{(iii)} Let $R$ be a relation between $X$ and $Y$ and $S$ a relation between $Y$ and $Z$. Then $(S \circ R)^{\op} = R^{\op} \circ S^{\op}$.
\wde{5.1.21 (Total) order} An \textit{order} (or \textit{order relation} or \textit{ordering}) on a set $X$ is a relation on $X$ that is reflexive, transitive, and antisymmetric. A set together with an order on it is called an \textit{ordered set}. \textbf{Def 5.1.23} It is \textit{total} when for all $x, y \in X$, we have $x \leq y$ or $y \leq x$.
