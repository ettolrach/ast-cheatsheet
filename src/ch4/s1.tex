\wde{4.1.1 Subset inclusion} For an injection $i \colon A \to X$, we write $A \subseteq X$ \textit{with inclusion} $i$. We often write the inclusion using a hooked arrow, $i \colon A \hookrightarrow X$.
\wde{4.1.7 X-element} Let $X$ be a set, let $A \subseteq X$ with inclusion $i$, and let $x \in X$. If there exists $a \in A$ such that $i(a) = x$ then we write $x \in_X A$ and call $x$ an \textit{$X$-element} of $A$. Strictly speaking, we should write $x \in_X (A \xrightarrow{i} X)$.
\wl{4.1.11} Let $f \colon X \to Y$ be a function and $y \in Y$. Then for all $x \in X$, $x \in_X f^{-1}(x) \iff f(x) = y$.
\wl{4.1.12} Let $X$ be a set and $A \subseteq X$. For $x \in X$, $x \in_X A \iff \chi_A(x) = \mathsf{T}$.
\wde{4.1.14} Let $X$ be a set and $A, B \subseteq X$. We write $A \subseteq_X B$ if for all $x \in X$, $X \in_X A \implies x \in_X B$.
\wl{4.1.17} Let $X$ be a set. Let $A \xhookrightarrow{i} X$ and $B \xhookrightarrow{j} X$ be subsets of $X$. The following are equivalent: \textbf{(i)} $A \subseteq_X B$; \textbf{(ii)} $\chi_A(x) = \mathsf{T} \implies \chi_B(x) = \mathsf{T}$, for $x \in X$; \textbf{(iii)} there exists a function $k \colon A \to B$ such that the triangle
% https://q.uiver.app/#q=WzAsMyxbMCwwLCJBIl0sWzIsMCwiQiJdLFsxLDEsIlgiXSxbMCwyLCJpIiwyXSxbMCwxLCJrIl0sWzEsMiwiaiJdXQ==
\[\begin{tikzcd}[cramped,row sep=tiny]
	A && B \\
	& X
	\arrow["k", from=1-1, to=1-3]
	\arrow["i"', from=1-1, to=2-2]
	\arrow["j", from=1-3, to=2-2]
\end{tikzcd}\]
commutes.
\wl{4.1.18} Let $j \colon B \to X$ be an injection, and let $k, k' \colon A \to B$ be functions such that $j \circ k = j \circ k'$. Then $k = k'$.
\wpr{4.1.20} Let $X$ be a set and $A, B \subseteq X$. The following are equivalent: \textbf{(i)} $A \cong_X B$; \textbf{(ii)} $\chi_A = \chi_B$; \textbf{(iii)} $A \subseteq_X B$ and $B \subseteq_X A$; \textbf{(iiii)} $x \in_X A \iff x \in_X B$, for $x \in X$.
