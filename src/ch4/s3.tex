\wde{4.3.2 Intersection} Let $X$ be a set and $A, B \subseteq X$. Their \textit{intersection} $A \cap B$ is the subset of $X$ whose characteristic function is the composite $X \xrightarrow{(\chi_A, \chi_B))} \mathbf{2} \times \mathbf{2} \xrightarrow{\wedge} \mathbf{2}$.
\wl{4.3.3} Let $X$ be a set and $A, B \subseteq X$. Then for $x \in X$, $x \in_X A \cap B \iff (x \in_X A \text{ and } x \in_X B)$.
\wde{4.3.4 Union} Let $X$ be a set and $A, B \subseteq X$. Their \textit{union} $A \cup B$ is the subset of $X$ whose characteristic function is the composite $X \xrightarrow{(\chi_A, \chi_B)} \mathbf{2} \times \mathbf{2} \xrightarrow{\vee} \mathbf{2}$.
\wl{4.3.5} Let $X$ be a set and $A, B \subseteq X$. Then for $x \in X$, $x \in_X A \cup B \iff (x \in_X A \text{ or } x \in_X B)$.
\wde{4.3.6 Complement} Let $X$ be a set and $A \subseteq X$. Its \textit{complement} $X \setminus A$ is the subset of $X$ whose characteristic function is the composite $X \xrightarrow{\chi_A} \mathbf{2} \xrightarrow{\neg} \mathbf{2}$.
\wl{4.3.7} Let $X$ be a set and $A \subseteq X$. Then for $x \in X$, $x \in_X X \setminus A \iff x \not \in_X A$.
\wl{4.3.9 Set laws} Let $A, B, C \subseteq X$ be sets. Then intersection and union are commutative, associative, and reflexive, they distribute over each other, and follow De Morgan's laws. Also $A \cap X = A$ and $A \cup \emptyset = A$. We also have complement laws: $A \cup (X \setminus A) = X$, $A \cap (X \setminus A) = \emptyset$, and $X \setminus (X \setminus A) = A$.
\wde{4.3.11 Relative complement} Let $X$ be a set and $A, B \subseteq X$. The \textit{relative complement} $B \setminus A$ is $B \cap (X \setminus A)$.
\wl{4.3.12} Let $X$ be a set and $A, B \subseteq X$. Then for $x \in X$, $x \in_B B \setminus A \iff x \in_X B \text{ and } x \not \in_X A$.
\wde{4.3.13} Let $X$ and $I$ be sets, and let $(A_i)_{i \in I} \in \mathscr{P}(X)^I$ be an $I$-indexed family of subsets of $X$. \textbf{(i)} The \textit{intersection} $\bigcap_{i \in I} A_i$ is the subset of $X$ whose characteristic function is the composite $X \xrightarrow{(\chi_{A_i})_{i \in I}} \mathbf{2}^I \xrightarrow{\bigwedge_I} \mathbf{2}$. \textbf{(ii)} The \textit{union} $\bigcup_{i \in I} A_i$ is the subset of $X$ whose characteristic function is the composite $X \xrightarrow{(\chi_{A_i})_{i \in I}} \mathbf{2}^I \xrightarrow{\bigvee_I} \mathbf{2}$.
\wl{4.3.14} Let $X$ and $I$ be ssets, and let $(A_i)_{i \in I} \in \mathscr{P}(X)^I$ be an $I$-indexed family of subsets of $X$. Then for $x \in X$, \textbf{(i)} $x \in_X \bigcap_{i \in I} A_i \iff x \in_X A_i$ for all $i \in I$; \textbf{(ii)} $x \in_X \bigcup_{i \in I} A_i \iff x \in_X A_i$ for some $i \in I$.
