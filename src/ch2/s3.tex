\wde{2.3.1 Terminal set} A set $T$ is \textit{terminal} if for all sets $X$, there is a unique function from $X$ to $T$. The definition of terminal set is the simplest example of a so-called \textit{universal property}.
\wax{2: Terminal set} There exists a terminal set, $\mathbf{1}$.
\wl{2.3.3} Any set isomorphic to a terminal set is terminal.
\wl{2.3.4} Any two terminal sets are isomorphic.
\wde{2.3.6 Element} An \textit{element} of a set $X$ is a function $\mathbf{1} \to X$. We write $x \in X$ to mean $x \colon \mathbf{1} \to X$.
\wl{2.3.7} \textbf{(i)} For all sets $X$ and elements $x \in X$, we have $\id_X(x) = x$. \textbf{(ii)} For all sets and functions $X \xrightarrow{f} Y \xrightarrow{g} Z$ and elements $x \in X$, we have $(g \circ f)(x) = g(f(x))$.
