\wde{Data} Some things are called \textit{sets}; for each set $X$ and set $Y$, some things called \textit{functions from $X$ to $Y$}, with functions $f$ from $X$ to $Y$ written as $f \colon X \to Y$; for each set$X$, set $Y$ and set $Z$, an operation called $composition$ assigning to each $f \colon X \to Y$ and $f \colon Y \to Z$ a function $g \circ f \colon X \to Z$; for each set $X$, a function $\text{id}_X \colon X \to X$, called an \textit{identity function}.
\wax{1: Associativity and identity} Composition of functions is associative: for all sets $W, X, Y, Z$ and functions $W \xrightarrow{f} X \xrightarrow{g} Y \xrightarrow{h} Z$, we have $h \circ g \circ f = h \circ (g \circ f)$. It also satisfies identity laws: for all sets $X$ and $Y$ and functions $f \colon X \to Y$, we have $f \circ \text{id}_X = f = id_Y \circ f$.
\wde{2.2.2 Inverse} Let $f \colon X \to Y$ be a function. An \textit{inverse} of $f$ is a function $g \colon X \to Y$ such that $g \circ f = id_X$ and $f \circ g = id_Y$.
\wl{2.2.4} A function has at most one inverse.
\wde{2.2.5 Invertible} A function is an \textit{isomorphism} or \textit{invertible} if it has an inverse.
\wl{2.2.6} \textbf{(i)} For all sets $X$, the identity function $\text{id}_X$ is an isomorphism, and $\text{id}_X^{-1} = \text{id}_X$. \textbf{(ii)} For all sets and functions $X \xrightarrow{f} Y \xrightarrow{g} Z$, if $f$ and $g$ are  isomorphisms then so is $g \circ f$, and $(g \circ f)^{-1} = f^{-1} \circ g^{-1}$. \textbf{(iii)}. For all functions $f \colon X \to Y$, if $f$ is an isomorphism then so is $f^{-1}$, and
$(f^{-1})^{-1} = f$.
\wde{2.2.8 Set isomorphism} Sets $X$ and $Y$ are \textit{isomorphic}, written as $X \cong Y$, if there exists an isomorphism from $X$ to $Y$.
\wl{2.2.9} \textbf{(i)} $X \cong X$ for all sets $X$. \textbf{(ii)} If $X \cong Y \cong Z$ then $X \cong Z$, for all sets $X$, $Y$, and $Z$. \textbf{(iii)} If $X \cong Y$ then $Y \cong X$, for all sets $X$ and $Y$.
