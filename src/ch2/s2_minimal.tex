\wax{1: Associativity and identity} Composition of functions is associative and have identity.
\wde{2.2.2 Inverse} Let $f \colon X \to Y$ be a function. An \textit{inverse} of $f$ is a function $g \colon X \to Y$ such that $g \circ f = id_X$ and $f \circ g = id_Y$.
\wl{2.2.4} A function has at most one inverse.
\wde{2.2.5 Invertible} A function is an \textit{isomorphism} or \textit{invertible} if it has an inverse.
\wl{2.2.6} We write the identity function $\text{id}_X$. Identities compose.
\wde{2.2.8 Set isomorphism} Sets $X$ and $Y$ are \textit{isomorphic}, written as $X \cong Y$, if there exists an isomorphism from $X$ to $Y$.
\wl{2.2.9} Isomorphisms are reflexive, and transitive.
