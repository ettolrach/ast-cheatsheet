\documentclass{article}

\usepackage{amsmath}
\usepackage{multicol}
\usepackage{titlesec}
\titleformat{\section}[runin]{\normalfont\Large\bfseries}{\thesection.}{6pt}{}[]
\titlespacing*{\section}
{0pt}{0pt}{0pt}


% Disable warnings
% chktex-file 1

\usepackage[margin=0.45in, portrait]{geometry} % may exceed print margins
\pagestyle{empty}

\usepackage[dvipsnames]{xcolor}

% Colour-blind friendly scheme
\definecolor{MyMagenta}{rgb}{0.91,0.16,0.54}
\definecolor{MyGreen}{rgb}{0.4,0.65,0.12}
\definecolor{MyPurple}{rgb}{0.46,0.44,0.7}
\definecolor{MyOrange}{rgb}{0.85,0.37,0.01}
\definecolor{MyEmerald}{rgb}{0.11,0.62,0.47}
\definecolor{MyYellow}{RGB}{230,171,2}

\colorlet{defc}{MyMagenta}
\colorlet{lemc}{MyGreen}
\colorlet{corc}{MyPurple}
\colorlet{propc}{MyOrange}
\colorlet{thmc}{MyEmerald}
\colorlet{axc}{MyYellow}

\colorlet{namec}{DarkOrchid}

\newcommand{\wde}[1]{\textcolor{defc}{\newline\textbf{DEF}} (\textcolor{namec}{\textit{#1}})}
\newcommand{\wl}[1]{\textcolor{lemc}{\newline\textbf{LEM}} (\textcolor{namec}{\textit{#1}})}
\newcommand{\wc}[1]{\textcolor{corc}{\newline\textbf{COR}} (\textcolor{namec}{\textit{#1}})}
\newcommand{\wpr}[1]{\textcolor{propc}{\newline\textbf{PROP}} (\textcolor{namec}{\textit{#1}})}
\newcommand{\wt}[1]{\textcolor{thmc}{\newline\textbf{THM}} (\textcolor{namec}{\textit{#1}})}
\newcommand{\wax}[1]{\textcolor{axc}{\newline\textbf{AX}} (\textcolor{namec}{\textit{#1}})}
\newcommand{\wpf}[1]{\textcolor{red}{\newline\textit{#1}}}  % Proof
\newcommand{\wa}[1]{\textcolor{blue}{\newline\textit{#1}}}  % Algorithm
\newcommand{\we}[1]{{\newline\textbf{EX}} (\textit{#1})}  % Example

\usepackage{amsmath}
\usepackage{amsthm}
\usepackage{amssymb}
\usepackage{stmaryrd}
\usepackage{mathrsfs}
\usepackage{mathtools}
\usepackage{tikz-cd}
\usepackage{quiver}

\newcommand{\cat}[1]{\mathscr{#1}}
\newcommand{\ob}[1]{\mathrm{ob}(\cat{#1})}
\DeclareMathOperator{\op}{op}
\newcommand{\map}[1]{\xrightarrow[]{#1}}
\newcommand{\maps}[2]{\begin{tikzcd}[ampersand replacement=\&,cramped, sep=scriptsize]
	{} \& {}
	\arrow["#1", shift left, from=1-1, to=1-2]
	\arrow["#2"', shift right, from=1-1, to=1-2]
\end{tikzcd}}

\newcommand{\isofuncs}[2]{\begin{tikzcd}[ampersand replacement=\&,cramped, sep=scriptsize]
	{} \& {}
	\arrow["#1", shift left, from=1-1, to=1-2]
	\arrow["#2", shift left, from=1-2, to=1-1]
\end{tikzcd}}

\newcommand{\ladj}[2]{\begin{tikzcd}[ampersand replacement=\&,cramped, sep=scriptsize]
	{}  \& {}
	\arrow[""{name=0, anchor=center, inner sep=0}, "#1", shift left=2, from=1-1, to=1-2]
	\arrow[""{name=1, anchor=center, inner sep=0}, "#2", shift left=2, from=1-2, to=1-1]
	\arrow["\dashv"{anchor=center, rotate=-90}, draw=none, from=0, to=1]
\end{tikzcd}}

\DeclareMathOperator{\Set}{\mathbf{Set}}
\DeclareMathOperator{\Cat}{\mathbf{Cat}}
\DeclareMathOperator{\CAT}{\mathbf{CAT}}
\newcommand{\adj}[0]{\dashv}
\newcommand{\comma}[0]{\downarrow}
\DeclareMathOperator{\Hom}{Hom}
\DeclareMathOperator{\iso}{\xrightarrow[]{\sim}}
\newcommand{\scat}[1]{\mathbf{#1}}
\DeclareMathOperator{\Cone}{Cone}
\DeclareMathOperator{\limcat}{\lim_{\leftarrow \scat{I}}}

\DeclareMathOperator{\id}{id}
\DeclareMathOperator{\fst}{pr_1}
\DeclareMathOperator{\snd}{pr_2}
\DeclareMathOperator{\N}{\mathbb{N}}
\DeclareMathOperator{\im}{im}
\DeclareMathOperator{\Pow}{\mathscr{P}}


%\fontsize{2}

\begin{document}
  \begin{multicols}{2}
    %
    %  CHAPTER 2: The axioms, part one
    %
    \setcounter{section}{1}
    \noindent\section{The axioms, part one}
    % 2.2 Associativity and identities
    \wde{Data} Some things are called \textit{sets}; for each set $X$ and set $Y$, some things called \textit{functions from $X$ to $Y$}, with functions $f$ from $X$ to $Y$ written as $f \colon X \to Y$; for each set$X$, set $Y$ and set $Z$, an operation called $composition$ assigning to each $f \colon X \to Y$ and $f \colon Y \to Z$ a function $g \circ f \colon X \to Z$; for each set $X$, a function $\text{id}_X \colon X \to X$, called an \textit{identity function}.
\wax{1: Associativity and identity} Composition of functions is associative: for all sets $W, X, Y, Z$ and functions $W \xrightarrow{f} X \xrightarrow{g} Y \xrightarrow{h} Z$, we have $h \circ g \circ f = h \circ (g \circ f)$. It also satisfies identity laws: for all sets $X$ and $Y$ and functions $f \colon X \to Y$, we have $f \circ \text{id}_X = f = id_Y \circ f$.
\wde{2.2.2 Inverse} Let $f \colon X \to Y$ be a function. An \textit{inverse} of $f$ is a function $g \colon X \to Y$ such that $g \circ f = id_X$ and $f \circ g = id_Y$.
\wl{2.2.4} A function has at most one inverse.
\wde{2.2.5 Invertible} A function is an \textit{isomorphism} or \textit{invertible} if it has an inverse.
\wl{2.2.6} \textbf{(i)} For all sets $X$, the identity function $\text{id}_X$ is an isomorphism, and $\text{id}_X^{-1} = \text{id}_X$. \textbf{(ii)} For all sets and functions $X \xrightarrow{f} Y \xrightarrow{g} Z$, if $f$ and $g$ are isomorphisms then so is $g \circ f$, and $(g \circ f)^{-1} = f^{-1} \circ g^{-1}$. \textbf{(iii)}. For all functions $f \colon X \to Y$, if $f$ is an isomorphism then so is $f^{-1}$, and
$(f^{-1})^{-1} = f$.
\wde{2.2.8 Set isomorphism} Sets $X$ and $Y$ are \textit{isomorphic}, written as $X \cong Y$, if there exists an isomorphism from $X$ to $Y$.
\wl{2.2.9} \textbf{(i)} $X \cong X$ for all sets $X$. \textbf{(ii)} If $X \cong Y \cong Z$ then $X \cong Z$, for all sets $X$, $Y$, and $Z$. \textbf{(iii)} If $X \cong Y$ then $Y \cong X$, for all sets $X$ and $Y$.

    %
    % 2.3 The one-element set
    \wde{2.3.1 Terminal set} A set $T$ is \textit{terminal} if for all sets $X$, there is a unique function from $X$ to $T$. The definition of terminal set is the simplest example of a so-called \textit{universal property}.
\wax{2: Terminal set} There exists a terminal set, $\mathbf{1}$.
\wl{2.3.3} Any set iso. to a terminal set is terminal.
\wl{2.3.4} Any two terminal sets are isom.
\wde{2.3.6 Element} An \textit{element} of a set $X$ is a function $\mathbf{1} \to X$. We write $x \in X$ to mean $x \colon \mathbf{1} \to X$.
\wl{2.3.7} \textbf{(i)} For all sets $X$ and elements $x \in X$, we have $\id_X(x) = x$. \textbf{(ii)} For all sets and functions $X \xrightarrow{f} Y \xrightarrow{g} Z$ and elements $x \in X$, we have $(g \circ f)(x) = g(f(x))$.

    %
    % 2.4  A function is determined by its effect on elements
    \wax{3: A function is determined by its effect on elements} For all sets $X$ and $Y$ and functions $f, g \colon X \to Y$, if $f(x) = g(x)$ for all $x \in X$ then $f = g$.
\wl{2.4.1} A set is terminal if and only if it has exactly one element.

    %
    % 2.5 The empty set
    \wde{2.5.1 Emptiness} A set is \textit{empty} if it has no elements.
\wax{4: Empty set} There exists an empty set.

    %
    % 2.6 Products
    \wde{2.6.2} Let $X$ and $Y$ be sets. A \textit{product} of $X$ and $Y$ consists of sets and functions $X \xleftarrow{p} P \xrightarrow{q} Y$ with the following property: for all sets $A$ and functions $f \colon A \to X$ and $g \colon A \to Y$, ther eis a unique function $h \colon A \to P$ such that $o \circ h = f$ and $q \circ h = g$. We call $X \xleftarrow{p} P \xrightarrow{q} Y$ a \textit{product diagram} and $p$ and $q$ are called \textit{projections}.
\wax{5: Products} Let $X$ and $Y$ be sets. Then $X$ and $Y$ have a product.
\wl{2.6.6} Let $X \xleftarrow{p} P \xrightarrow{q} Y$ and $X \xleftarrow{p'} P' \xrightarrow{q'} Y$ be sets and functions, and suppose there exists an isomorphism $j \colon P \to P'$ such that the diagram
% The "\cong" label was added manually.
% https://q.uiver.app/#q=WzAsNCxbMSwwLCJQIl0sWzAsMSwiWCJdLFsyLDEsIlkiXSxbMSwyLCJQJyJdLFswLDMsImoiXSxbMywxLCJwJyJdLFswLDEsInAiLDJdLFswLDIsInEiLDJdLFszLDIsInEnIiwyXV0=
\[\begin{tikzcd}[cramped, row sep=tiny]
	& P \\
	X && Y \\
	& {P'}
	\arrow["p"', from=1-2, to=2-1]
	\arrow["q", from=1-2, to=2-3]
	\arrow["j", "\cong"', from=1-2, to=3-2]
	\arrow["{p'}", from=3-2, to=2-1]
	\arrow["{q'}"', from=3-2, to=2-3]
\end{tikzcd}\]
commutes. If one of the first two diagrams is a product then so is the other.
\wl{2.6.8} Let $X \xleftarrow{p} P \xrightarrow{q} Y$ and $X \xleftarrow{p'} P' \xrightarrow{q'} Y$ be product diagrams. Then there is a unique isomorphism $j \colon P \to P'$ ssuch that the diagram [same as the earlier diagram in 2.6.6] commutes. In particular, $P \cong P'$.
\wde{no number} We write $X \xleftarrow{\fst^{X, Y}} X \times Y \xrightarrow{\snd^{X, Y}}$ for the projections. We often abbreviate to just $\fst$ and $\snd$.
\wde{no number} Given sets and functions $X \xleftarrow{f} X \times Y \xrightarrow{g}$, we write $(f, g) \colon A \to X \times Y$ for the unique $A \to X \times Y$ satisfying $\fst \circ (f, g) = f$, $\snd \circ (f, g) = g$.
\wl{2.6.10} Take sets and functions $X \xleftarrow{f} A \xrightarrow{g} Y$. Then $(f, g)(a) = (f(a), g(a))$ for all $a \in A$.
\we{2.6.11 i.} For every set $X$, the functions $X \xleftarrow{\id_X} X \xrightarrow{\id_X} X$ give rise to the \textit{diagonal function} $\triangle_X = (\id_X, \id_X) \colon X \to X \times X$. By lemma 2.6.10, $\triangle_X(x) = (x, x)$ for all $x \in X$.
\wl{2.6.13} Let $X \xrightarrow{f} X'$ anad $Y \xrightarrow{g} Y'$ be functions. Then there is a unique function $f \times g \colon X \times Y \to X' \times Y'$ such that $(f \times g)(x, y) = (f(x), g(y))$ for all $x \in X$ and $y \in Y$.
\wpr{2.6.15} \textbf{(i)} $X \times Y \cong Y \times X$ for all sets $X$ and $Y$. \textbf{(ii)} $X \times \textbf{1} \cong X$ for all sets $X$. \textbf{(iii)} $(X \times Y) \times Z \cong X \times (Y \times Z)$ for all sets $X, Y, Z$.

    %
    % 2.7 Sets of functions
    \wde{2.7.3 Function set} Let $X$ and $Y$ be sets. A \textit{function set} from $X$ to $Y$ is a set $F$ together with a function $e \colon F \times X \to Y$, with the following property: for all sets $A$ and functions $q \colon A \times X \to Y$, there is a unique function $\bar{q} \colon A \to F$ such that $q(a, x) = e(\bar{q}(a), x)$ for all $a \in A$ and $x \in X$. A function $e$ is called \textit{evaluation}.
\wax{6: Function sets} Let $X$ and $Y$ be sets. Then there exists a function set form $X$ to $Y$.
\wl{2.7.5} Let $X$ and $Y$ be sets. Take sets $F$ and $F'$ and functions $F \times X \xrightarrow{e} Y$, $F' \times X \xrightarrow{e'} Y$, and suppose there exists an isomorphism $j \colon F \to F'$ such that the diagram
% https://q.uiver.app/#q=WzAsMyxbMCwwLCJGIFxcdGltZXMgWCJdLFswLDIsIkYnIFxcdGltZXMgWCJdLFsyLDEsIlkiXSxbMCwxLCJqIFxcdGltZXMgXFx0ZXh0e2lkfV9YIiwyXSxbMCwyLCJlIl0sWzEsMiwiZSciLDJdXQ==
\[\begin{tikzcd}[cramped,row sep=tiny]
	{F \times X} \\
	&& Y \\
	{F' \times X}
	\arrow["e", from=1-1, to=2-3]
	\arrow["{j \times \text{id}_X}"', "\cong", from=1-1, to=3-1]
	\arrow["{e'}"', from=3-1, to=2-3]
\end{tikzcd}\]
commutes. Then $F$ together with $e$ is a function set from $X$ to $Y$ if and only if $F'$ together with $e'$ is.
\wl{2.7.6} Let $X$ and $Y$ be sets. Let $F$ together with $e \colon F \times X \to Y$ be a function set form $X$ to $Y$, and let $F'$ together with $e' \colon F' \times X \to Y$ be another function set from $X$ to $Y$. Then there is a unique isomorphism $j \colon F \to F'$ such that the diagram [same as the earlier diagram in 2.7.5] commutes. In particular, $F \cong F'$.
\wde{no number} We write $Y^X$ for the function set from $X$ to $Y$, and $\text{ev}_{X, Y} \colon Y^X \times X \to Y$ (or sometimes just $\text{ev}$) for the evaluation function. Given a function $q \colon A \times X \to Y$, we write $\bar{q}$ for the corresponding function $A \to Y^X$. For a function $f \colon X \to Y$, we write $\bar{f}$ for the corresponding element of $Y^X$.
\wde{2.7.8 X-indexed family} Let $X$ and $Y$ be sets. An \textit{$X$-indexed family of elements of $Y$} is a function from $X$ to $Y$.
\we{2.7.9} Let $X$ be $\N$. Then $Y^X = Y^{\N}$ can be thought of as the set of sequences $(y_n)_{n \in \N}$ of elements of $Y$.
\wl{2.7.10} \textbf{(i)} Let $X$ be a set and $v \colon Y \to Y'$ a function. Then there is a unique function $v^X \colon Y^X \to Y'^X$ such that $v^X(f) = v \circ f$ for all $f \in Y^X$. \textbf{(ii)} Let $Y$ be a set and $u \colon X \to X'$ a function. Then there is a unique function $Y^u \colon Y^{X'} \to Y^X$ such that $Y^u(f') = f' \circ u$ for all $f' \in Y^{X'}$.
\wpr{2.7.12} \textbf{(i)} $(Y \times Z)^X \cong Y^X \times Z^X$ for all sets $X$, $Y$, $Z$, and $\mathbf{1}^X \cong \mathbf{1}$ for all sets $X$. \textbf{(ii)} $Z^{X \times Y} \cong (Z^Y)^X$ for all sets $X$, $Y$, $Z$, and $Y^{\mathbf{1}} \cong Y$ for all sets $Y$.

    %
    %
    %  CHAPTER 3: The axioms, part two
    %
    \noindent\section{The axioms, part two}
    % 3.1 Fibres
    \wde{3.1.4 Fibre} Let $f \colon X \to Y$ be a function and $y \in Y$. A \textit{fibre} of $f$ over $y$ is a set $U$ together with a function $i \colon U \to X$, such that $f(i(u)) = y$ for all $u \in U$ and the following property holds: for all sets $A$ and functions $q \colon A \to X$ such that $f(q(a)) = y$ for all $a \in A$, there is a unique function $\bar{q} \colon A \to U$ such that $q = i \circ \bar{q}$. We refer to $i$ as the \textit{inclusion} of the fibre.
\we{3.1.5} Special case $A = \mathbf{1}$ is particularly important. It tells us that for all $x \in X$ satisfying $f(x) = y$, there is a unique $u \in U$ such that $x = i(u)$.
\wax{7: Fibres} For every function $f \colon X \to Y$ and element $y \in Y$, there exists a fibre of $f$ over $y$.
\wl{3.1.6} Let $f \colon X \to Y$ be a function and $y \in Y$. Take sets and functions $U \xrightarrow{i} X$ and $U' \xrightarrow{i'} X$, and suppose there exists an isomorphism $j \colon U \to U'$ such that the diagram
% https://q.uiver.app/#q=WzAsMyxbMCwwLCJVIl0sWzIsMSwiWCJdLFswLDIsIlUnIl0sWzAsMSwiaSJdLFsyLDEsImknIiwyXSxbMCwyLCJqIiwyXV0=
\[\begin{tikzcd}[cramped,row sep=tiny]
	U \\
	&& X \\
	{U'}
	\arrow["i", from=1-1, to=2-3]
	\arrow["j"', "\cong", from=1-1, to=3-1]
	\arrow["{i'}"', from=3-1, to=2-3]
\end{tikzcd}\]
commutes. Then $U \xrightarrow{i} X$ is a fibre of $f$ over $y$ if and only if $U' \xrightarrow{i'} X$ is.
\wde{no number} The fibre of $f$ over $y$ is given the name $f^{-1}(y)$. The inclusion $f^{-1}(y) \to X$ is usually nameless or given the symbol $\hookrightarrow$.
\wde{3.1.8} Let $S$ and $T$ be sets. A function $g \colon S \to T$ is: \textbf{(i)} \textit{injective} if for all $s, s' \in S$, $g(s) = g(s') \implies s = s'$; \textbf{(ii)} \textit{surjective} if for all $t \in T$, there exists $s \in S$ such that $g(s) = t$; \textbf{(iii)} \textit{bijective} if it is injective and surjective.
\wl{3.1.11} Let $f \colon X \to Y$ be a function and $y \in Y$. Then the inclusion $f^{-1}(y) \hookrightarrow X$ is injective.
    %
    % 3.2 Subsets
    \wde{3.2.1 Subset classifier} A \textit{subset classifier} is a set $\Omega$ together with an element $t \in \Omega$, with the following property: for all sets $A$ and $X$ and injections $i \colon A \to X$, $\exists!$ fn. $\chi \colon X \to \Omega$ such that $i \colon A \to X$ is a fibre of $\chi$ over $t$. In a diagram (left):\linebreak
% https://q.uiver.app/#q=WzAsNCxbMCwwLCJBIl0sWzEsMCwiXFxtYXRoYmZ7MX0iXSxbMSwxLCJcXE9tZWdhIl0sWzAsMSwiWCJdLFswLDFdLFsxLDIsInQiXSxbMCwzLCJpIiwyXSxbMywyLCJcXGNoaSIsMix7InN0eWxlIjp7ImJvZHkiOnsibmFtZSI6ImRhc2hlZCJ9fX1dXQ==
% https://q.uiver.app/#q=WzAsNCxbMCwwLCJBIl0sWzEsMCwiXFxtYXRoYmZ7MX0iXSxbMSwxLCJcXG1hdGhiZnsyfSJdLFswLDEsIlgiXSxbMCwxXSxbMSwyLCJcXG1hdGhzZntUfSJdLFswLDMsImkiLDJdLFszLDIsIlxcY2hpX2kiLDIseyJzdHlsZSI6eyJib2R5Ijp7Im5hbWUiOiJkYXNoZWQifX19XV0=
\[\begin{tikzcd}[cramped]
    A & {\mathbf{1}} \\
    X & \Omega
    \arrow[from=1-1, to=1-2]
    \arrow["i"', from=1-1, to=2-1]
    \arrow["t", from=1-2, to=2-2]
    \arrow["\chi"', dashed, from=2-1, to=2-2]
\end{tikzcd}
\begin{tikzcd}[cramped]
    A & {\mathbf{1}} \\
    X & {\mathbf{2}}
    \arrow[from=1-1, to=1-2]
    \arrow["i"', from=1-1, to=2-1]
    \arrow["{\mathsf{T}}", from=1-2, to=2-2]
    \arrow["{\chi_i}"', dashed, from=2-1, to=2-2]
\end{tikzcd}\]

\wax{8: Subsets} There exists a subset classifier.
\wl{3.2.2} Let $\Omega$ and $\Omega'$ be sets, and let $t \in \Omega$ and $t' \in \Omega'$. Suppose $\exists$ an isomorphism $j \colon \Omega \to \Omega'$ such that $j(t) = t'$. Then $\Omega$ together with $t$ is a subset classifier iff $\Omega'$ together with $t'$ is.
\wl{3.2.3} Let $\Omega$ together with $t \in \Omega$ be a subset classifier, and let $\Omega'$ together with $t' \in \Omega'$ be another subset classifier. Then $\exists!$ isomorphism $j \colon \Omega \to \Omega'$ such that $j(t) = t'$.
\wde{n.n.} We denote the subset classifier by $\mathbf{2}$ and its distinguishing element by $\textsf{T}$. Given an injection $A \xrightarrow{i} X$, we write $\chi_i$ for the unique fn. $X \to \mathbf{2}$ whose fibre over $\mathsf{T}$ is $A \xrightarrow{i} X$. It's the characteristic fn. of $i$. In a diagram: [above right].
\wl{3.2.5} Let $i \colon A \to X$ be an injection and $x \in X$. Then $\chi_i(x) = \mathsf{T} \iff$ there exists $a \in A$ such that $i(a) = x$. (Since $i$ is injective, the $a$ here is necessarily unique.)
\wde{n.n.} For an injection $A \xrightarrow{i} X$, we sometimes write $\chi_i$ as $\chi_A$ and call it the characteristic fn. of $A$.
\wde{3.2.8 Subset} Let $X$ be a set. A \textit{subset} of $X$ is a fn. $X \to \mathbf{2}$.
\wde{3.2.9 Power set} Let $X$ be a set. The \textit{power set} of $X$ is $\mathbf{2}^X$, also written as $\mathscr{P}(X)$.
\wde{3.2.12 Isomorphism over a set} Let $X$ be a set. Two fn.s $A \xrightarrow{i} X$ and $A' \xrightarrow{i'} X$ are \textit{isomorphic over X} if there exists an isomorphism $j \colon A \to A'$ such that the diagram
% https://q.uiver.app/#q=WzAsMyxbMCwwLCJBIl0sWzIsMCwiQSciXSxbMSwxLCJYIl0sWzAsMSwiaiJdLFsxLDIsImknIl0sWzAsMiwiaSIsMl1d
\[\begin{tikzcd}[cramped,row sep=tiny]
    A && {A'} \\
    & X
    \arrow["j", "\cong"', from=1-1, to=1-3]
    \arrow["i"', from=1-1, to=2-2]
    \arrow["{i'}", from=1-3, to=2-2]
\end{tikzcd}\]
commutes. We then write $(A \xrightarrow{i} X) \cong (A' \xrightarrow{i'} X)$ or, less formally, $A \cong_X A'$.
\wl{3.2.16} Let $X$ be a set, and let $A \xrightarrow{i} X$ and $A' \xrightarrow{i'} X$ be injections. Then $(A \xrightarrow{i} X) \cong (A' \xrightarrow{i'} X) \iff \chi_i = \chi_{i'}$. Or in more casual notation, $A \cong_X A' \iff \chi_A = \chi_{A'}$.
\wl{n.n.} There is a one-to-one correspondence between the fn.s $X \to \mathbf{2}$ and isomorphism classes of injections into $X$, where 'isomorphism' means isomorphism over $X$.
\wpr{3.2.18} Let $i \colon A \to X$ be a fn.. The following are equivalent: \textbf{(i)} $i$ is an isomorphism; \textbf{(ii)} $i$ is a bijection; \textbf{(iii)} $i$ is an injection and $\chi_i(x) = \mathsf{T}$ for all $x \in X$.
\wde{3.2.21 Initial set} A set $I$ is \textit{initial} if for all sets $X$, there is a unique fn. from $I$ to $X$.
\wpr{3.2.22} A set is empty iff it is initial.
\wc{3.2.23} Any two empty sets are isomorphic.
    %
    % 3.3 The natural numbers
    \linebreak\textbf{WARNING:} In this course, $0$ is a natural number.
\wde{3.3.2 A natural number system} A \textit{natural number system} is a set $N$ together with an element $z \in N$ and a function $\sigma \colon N \to N$, with the following property: for all sets $X$, elements $a \in X$ and functions $r \colon X \to X$, there is a unique function $x \colon N \to X$ such that $x(z) = a$ and $x(\sigma(n)) = r(x(n))$ for all $n \in N$. In a diagram:
% https://q.uiver.app/#q=WzAsNixbMCwwLCJcXG1hdGhiZnsxfSJdLFsxLDAsIk4iXSxbMiwwLCJOIl0sWzIsMSwiWCJdLFsxLDEsIlgiXSxbMCwxLCJcXG1hdGhiZnsxfSJdLFswLDEsInoiXSxbMCw1LCJcXHRleHR7aWR9IiwyXSxbNSw0LCJhIiwyXSxbNCwzLCJyIiwyXSxbMSwyLCJcXHNpZ21hIl0sWzIsMywieCIsMCx7InN0eWxlIjp7ImJvZHkiOnsibmFtZSI6ImRhc2hlZCJ9fX1dLFsxLDQsIngiLDAseyJzdHlsZSI6eyJib2R5Ijp7Im5hbWUiOiJkYXNoZWQifX19XV0=
\[\begin{tikzcd}[cramped]
	{\mathbf{1}} & N & N \\
	{\mathbf{1}} & X & X
	\arrow["z", from=1-1, to=1-2]
	\arrow["{\text{id}}"', from=1-1, to=2-1]
	\arrow["\sigma", from=1-2, to=1-3]
	\arrow["x", dashed, from=1-2, to=2-2]
	\arrow["x", dashed, from=1-3, to=2-3]
	\arrow["a"', from=2-1, to=2-2]
	\arrow["r"', from=2-2, to=2-3]
\end{tikzcd}\]
\wax{9: Natural numbers} There exists a natural number system.
\wl{3.3.3} Let $N$ and $N'$ be sets, let $z \in N$ and $z' \in N'$, andlet $\sigma \colon N \to N$ and $\sigma' \colon N' \to N'$. Suppose there exists an isomorphism $j \colon N \to N'$ such that the diagram
% https://q.uiver.app/#q=WzAsNixbMCwwLCJcXG1hdGhiZnsxfSJdLFsxLDAsIk4iXSxbMiwwLCJOIl0sWzIsMSwiTiciXSxbMSwxLCJOJyJdLFswLDEsIlxcbWF0aGJmezF9Il0sWzAsMSwieiJdLFswLDUsIlxcdGV4dHtpZH0iLDJdLFs1LDQsInonIiwyXSxbNCwzLCJcXHNpZ21hJyIsMl0sWzEsMiwiXFxzaWdtYSJdLFsyLDMsImoiLDAseyJzdHlsZSI6eyJib2R5Ijp7Im5hbWUiOiJkYXNoZWQifX19XSxbMSw0LCJqIiwwLHsic3R5bGUiOnsiYm9keSI6eyJuYW1lIjoiZGFzaGVkIn19fV1d
\[\begin{tikzcd}[cramped]
	{\mathbf{1}} & N & N \\
	{\mathbf{1}} & {N'} & {N'}
	\arrow["z", from=1-1, to=1-2]
	\arrow["{\text{id}}"', from=1-1, to=2-1]
	\arrow["\sigma", from=1-2, to=1-3]
	\arrow["j", "\cong"', dashed, from=1-2, to=2-2]
	\arrow["j", "\cong"', dashed, from=1-3, to=2-3]
	\arrow["{z'}"', from=2-1, to=2-2]
	\arrow["{\sigma'}"', from=2-2, to=2-3]
\end{tikzcd}\]
commutes. Then $N$ together with $z$ and $\sigma$ is a natural number system if and only if $N'$ together with $z'$ and $\sigma'$ is.
\wl{3.3.4} Let $N$ together with $z \in N$ and $\sigma \colon N \to N$ be a natural number system, and let $N'$ together with $z' \in N'$ and $\sigma' \colon N' \to N'$ be another natural number system. Then there is a unique isomorphism $j \colon N \to N'$ such that the diagram [same diagram as above in lemma 3.3.3] commutes.
\wde{no number} We write the natural number system as $\N$ together with $0 \in \N$ and $s \colon \N \to \N$. The function $s$ is called the \textit{successor} function.
    %
    % 3.4 The axiom of choice
    \wde{3.4.1 Section} A \textit{section} of a function $f \colon X \to Y$ is a function $i \colon Y \to X$ such that $f \circ i = \id_Y$. Thus, it is another word for \textit{right inverse}.
\wax{10: Axiom of choice} Every surjection has a section.
    %
    %
    %  CHAPTER 4: Subsets
    %
    \noindent\section{Subsets}
    % 4.1 Elements and subsets
    \wde{4.1.1 Subset inclusion} For an injection $i \colon A \to X$, we write $A \subseteq X$ \textit{with inclusion} $i$. We often write the inclusion using a hooked arrow, $i \colon A \hookrightarrow X$.
\wde{4.1.7 X-element} Let $X$ be a set, let $A \subseteq X$ with inclusion $i$, and let $x \in X$. If there exists $a \in A$ such that $i(a) = x$ then we write $x \in_X A$ and call $x$ an \textit{$X$-element} of $A$. Strictly speaking, we should write $x \in_X (A \xrightarrow{i} X)$.
\wl{4.1.11} Let $f \colon X \to Y$ be a function and $y \in Y$. Then for all $x \in X$, $x \in_X f^{-1}(x) \iff f(x) = y$.
\wl{4.1.12} Let $X$ be a set and $A \subseteq X$. For $x \in X$, $x \in_X A \iff \chi_A(x) = \mathsf{T}$.
\wde{4.1.14} Let $X$ be a set and $A, B \subseteq X$. We write $A \subseteq_X B$ if for all $x \in X$, $X \in_X A \implies x \in_X B$.
\wl{4.1.17} Let $X$ be a set. Let $A \xhookrightarrow{i} X$ and $B \xhookrightarrow{j} X$ be subsets of $X$. The following are equivalent: \textbf{(i)} $A \subseteq_X B$; \textbf{(ii)} $\chi_A(x) = \mathsf{T} \implies \chi_B(x) = \mathsf{T}$, for $x \in X$; \textbf{(iii)} there exists a function $k \colon A \to B$ such that the triangle
% https://q.uiver.app/#q=WzAsMyxbMCwwLCJBIl0sWzIsMCwiQiJdLFsxLDEsIlgiXSxbMCwyLCJpIiwyXSxbMCwxLCJrIl0sWzEsMiwiaiJdXQ==
\[\begin{tikzcd}[cramped,row sep=tiny]
	A && B \\
	& X
	\arrow["k", from=1-1, to=1-3]
	\arrow["i"', from=1-1, to=2-2]
	\arrow["j", from=1-3, to=2-2]
\end{tikzcd}\]
commutes.
\wl{4.1.18} Let $j \colon B \to X$ be an injection, and let $k, k' \colon A \to B$ be functions such that $j \circ k = j \circ k'$. Then $k = k'$.
\wpr{4.1.20} Let $X$ be a set and $A, B \subseteq X$. The following are equivalent: \textbf{(i)} $A \cong_X B$; \textbf{(ii)} $\chi_A = \chi_B$; \textbf{(iii)} $A \subseteq_X B$ and $B \subseteq_X A$; \textbf{(iiii)} $x \in_X A \iff x \in_X B$, for $x \in X$.

    %
    % 4.2 Truth values
    \wde{4.2.1 $\mathsf{F}$} The element $\mathsf{F}$ of $\mathbf{2}$ (pronounced `false') is $\chi_{\emptyset \to \mathbf{1}}$.
\wt{4.2.2} The subset classifier $\mathbf{2}$ has exactly two elements, $\mathsf{T}$ and $\mathsf{F}$. That is, $\mathsf{T} \neq \mathsf{F}$, and every element of $\mathbf{2}$ is equal to $\mathsf{T}$ or $\mathsf{F}$. (These are called \textit{truth values}.)
\wc{4.2.3} Let $X$ be a set, and let $A \subseteq X$. Then for $x \in X$,
\begin{equation*}
	\begin{cases}
		\mathsf{T} & \text{if } x \in_X A,\\
		\mathsf{F} & \text{otherwise.}
	\end{cases}
\end{equation*}
\wde{4.2.7 Negation} The \textit{negation operator} $\neg \colon \mathbf{2} \to \mathbf{2}$ is $\chi_{\{\mathsf{F}\}}$, the characteristic function of $\{\mathsf{F}\} \subseteq \mathbf{2}$.
\wl{4.2.8} $\neg \mathsf{T} = \mathsf{F}$ and $\neg \mathsf{F} = \mathsf{T}$.
\wl{4.2.9} $\neg \circ \neg = \id_{\mathbf{2}}$.
\wde{4.2.11 Conjunction} The \textit{conjunction operator} $\wedge \colon \mathbf{2} \times \mathbf{2} \to \mathbf{2}$ is $\chi_{\{(\mathsf{T}, \mathsf{T})\}}$, the characteristic function of $\{(\mathsf{T}, \mathsf{T})\} \subseteq \mathbf{2} \times \mathbf{2}$.
\wl{4.2.13} $\alpha \wedge \beta = \mathsf{T} \iff \alpha = \beta = \mathsf{T}$. Otherwise, $\alpha \wedge \beta = \mathsf{F}$.
\wde{4.2.15 Disjunction} The \textit{disjunction operator} $\vee \colon \mathbf{2} \times \mathbf{2} \to \mathbf{2}$ is the composite function $\mathbf{2} \times \mathbf{2} \xrightarrow{\neg \times \neg} \mathbf{2} \times \mathbf{2} \xrightarrow{\wedge} \xrightarrow{\neg} \mathsf{2}$.
\wl{4.2.16} $\alpha \vee \beta = \mathsf{F} \iff \alpha = \beta = \mathsf{F}$. Otherwise, $\alpha \wedge \beta = \mathsf{T}$.
\wde{4.2.17 $I$-fold conjuction operator} For a set $I$, the \textit{$I$-fold conjunction operator} $\bigwedge_I \colon \mathbf{2}^I \to \mathbf{2}$ is $\chi_{\{(\mathsf{T}_{i \in I})\}}$, the characteristic function of $\{(\mathsf{T}_{i \in I})\} \subseteq \mathbf{2}^I$. For a family $(\alpha_i)_{i \in I} \in \mathbf{2}^I$, we write $\bigwedge_I((\alpha_i)_{i \in I})$ as $\bigwedge_{i \in I} \alpha_i$.
\wl{4.2.18} Let $I$ be a set and $(\alpha_i)_{i \in I} \in \mathbf{2}^I$. Then $\bigwedge_{i \in I} \alpha_i = \mathsf{T}$ if and only if $\alpha_i = \mathsf{T}$ for all $i \in I$.
\wde{4.2.19 $I$-fold disjuction operator} Let $I$ be a set. The \textit{$I$-fold disjunction operator} $\bigvee_I \colon \mathbf{2}^I \to \mathbf{2}$ is the composite $\mathbf{2}^I \xrightarrow{\neg^I} \mathbf{2}^I \xrightarrow{\bigwedge_I} \mathbf{2} \xrightarrow{\neg} \mathbf{2}$. We write $\bigvee_I((\alpha_i)_{i \in I})$ as $\bigvee_{i \in I} \alpha_i$. So, $\bigvee_{i \in I} \alpha_i = \neg \bigwedge_{i \in I} \neg \alpha_i$.
\wl{4.2.20} Let $I$ be a set and $(\alpha_i)_{i \in I} \in \mathbf{2}^I$. Then $\bigvee_{i \in I} \alpha_i = \mathsf{T}$ if and only if $\alpha_i = T$ for some $i \in I$.
    %
    % 4.3 Operations on subsets
    \wde{4.3.2 Intersection} Let $X$ be a set and $A, B \subseteq X$. Their \textit{intersection} $A \cap B$ is the subset of $X$ whose characteristic function is the composite $X \xrightarrow{(\chi_A, \chi_B))} \mathbf{2} \times \mathbf{2} \xrightarrow{\wedge} \mathbf{2}$.
\wl{4.3.3} Let $X$ be a set and $A, B \subseteq X$. Then for $x \in X$, $x \in_X A \cap B \iff (x \in_X A \text{ and } x \in_X B)$.
\wde{4.3.4 Union} Let $X$ be a set and $A, B \subseteq X$. Their \textit{union} $A \cup B$ is the subset of $X$ whose characteristic function is the composite $X \xrightarrow{(\chi_A, \chi_B)} \mathbf{2} \times \mathbf{2} \xrightarrow{\vee} \mathbf{2}$.
\wl{4.3.5} Let $X$ be a set and $A, B \subseteq X$. Then for $x \in X$, $x \in_X A \cup B \iff (x \in_X A \text{ or } x \in_X B)$.
\wde{4.3.6 Complement} Let $X$ be a set and $A \subseteq X$. Its \textit{complement} $X \setminus A$ is the subset of $X$ whose characteristic function is the composite $X \xrightarrow{\chi_A} \mathbf{2} \xrightarrow{\neg} \mathbf{2}$.
\wl{4.3.7} Let $X$ be a set and $A \subseteq X$. Then for $x \in X$, $x \in_X X \setminus A \iff x \not \in_X A$.
\wl{4.3.9 Set laws} Let $X$ be a set and $A, B, C \subseteq X$. Then the following laws hold. \textbf{(i)} Intersection laws: $\cap$ is commutative, associative, and reflexive, as well as $A \cap X = A$. \textbf{(ii)} Union laws: $\cup$ is commutative, associative, and reflexive, as well as $A \cup \emptyset = A$. \textbf{(iii)} Distributive laws: $\cap$ distributes over $\cup$, $\cup$ distributes over $\cap$, and $A \cap \emptyset = A$ and $A \cup X = X$. \textbf{(iv)} De Morgan's laws: $X \setminus (A \cup B) = (X \setminus A) \cap (X \setminus B)$ and $X \setminus (A \cap B) = (X \setminus A) \cup (X \setminus B)$ \textbf{(v)} Complement laws: $A \cup (X \setminus A) = X$, $A \cap (X \setminus A) = \emptyset$, and $X \setminus (X \setminus A) = A$.
\wde{4.3.11 Relative complement} Let $X$ be a set and $A, B \subseteq X$. The \textit{relative complement} $B \setminus A$ is $B \cap (X \setminus A)$.
\wl{4.3.12} Let $X$ be a set and $A, B \subseteq X$. Then for $x \in X$, $x \in_B B \setminus A \iff x \in_X B \text{ and } x \not \in_X A$.
\wde{4.3.13} Let $X$ and $I$ be sets, and let $(A_i)_{i \in I} \in \mathscr{P}(X)^I$ be an $I$-indexed family of subsets of $X$. \textbf{(i)} The \textit{intersection} $\bigcap_{i \in I} A_i$ is the subset of $X$ whose characteristic function is the composite $X \xrightarrow{(\chi_{A_i})_{i \in I}} \mathbf{2}^I \xrightarrow{\bigwedge_I} \mathbf{2}$. \textbf{(ii)} The \textit{union} $\bigcup_{i \in I} A_i$ is the subset of $X$ whose characteristic function is the composite $X \xrightarrow{(\chi_{A_i})_{i \in I}} \mathbf{2}^I \xrightarrow{\bigvee_I} \mathbf{2}$.
\wl{4.3.14} Let $X$ and $I$ be ssets, and let $(A_i)_{i \in I} \in \mathscr{P}(X)^I$ be an $I$-indexed family of subsets of $X$. Then for $x \in X$, \textbf{(i)} $x \in_X \bigcap_{i \in I} A_i \iff x \in_X A_i$ for all $i \in I$; \textbf{(ii)} $x \in_X \bigcup_{i \in I} A_i \iff x \in_X A_i$ for some $i \in I$.

    %
    % 4.4 Images and preimages
    \linebreak\textbf{WARNING:} We write $f_*A$ for the image of $fA$ and $f^*B$ for the preimage $f^{-1}B$.
\wde{no number, Image} Let $f \colon X \to Y$ and $A \subseteq X$ with the inclusion $i$. The \textit{image of $A$ under $f$} is defined to be $f_*A = \bigcup_{a \in A} \{ fi(a) \} \subseteq Y$.
\wl{4.4.2} Let $f \colon X \to Y$ be a function and $A \subseteq X$. Then for $y \in Y$, $y \in_Y f_*A \iff y = f(x)$ for some $x \in X$ such that $x \in_X A$.
\we{4.4.3 (ii)} Let $f \colon X \to Y$. Consider the subset $X$ of $X$. Its image $f_*X \subseteq Y$ has the property that for $y \in Y$, $y \in_Y f_*X \iff y = f(x)$ for some $x \in X$. We call $f_*X$ the \textit{image of $f$} and write it as $\im f$.
\wl{4.4.4} Let $f \colon X \to Y$ be a function, let $I$ be a set, and let $(A_i)_{i \in I}$ be a family of subsets of $X$. Then $f_*(\bigcap_{i \in I} A_i) = \bigcap_{i \in I} f_* A_i$, $f_*(\bigcup_{i \in I} A_i) \subseteq \bigcup_{i \in I} f_* A_i$.
\wde{no number, Preimage} Let $f \colon X \to Y$ and a subset $B \subseteq Y$. The \textit{preimage of $B$ under $f$} is the subset $f^*B$ of $X$ whose characteristic function is the composite $X \xrightarrow{f} Y \xrightarrow{\chi_B} \mathbf{2}$.
\wl{4.4.7} Let $f \colon X \to Y$ be a function aand $B \subseteq Y$. Then for $x \in X$, $x \in_X f^*B \iff f(x) \in_Y B$.
\wl{4.4.9} Let $f \colon X \to Y$ be a function, let $I$ be a set, and let $(B_i)_{i \in I}$ be a family of subsets of $Y$. Then $f^*(\bigcup_{i \in I} B_i) = \bigcup_{i \in I} f^* B_i$, $f^*(\bigcap_{i \in I} B_i) = \bigcap_{i \in I} f^* B_i$.
\wl{4.4.10} Let $f \colon X \to Y$ be an invertible function and $B \subseteq Y$. Then $(f^{-1})_*B = f^*B$.
\wl{4.4.12} Let $f \colon X \to Y$ be a function, let $A \subseteq X$, and let $B \subseteq Y$. Then $f_*A \subset_Y B \iff A \subseteq_X f^*B$.
\wl{4.4.13} Let $f: X \to Y$. Then $A \subseteq_X f^*f_*A$ for all $A \subseteq X$, and $f_*f^*B \subseteq_Y B$ for all $B \subseteq Y$.

    %
    %
    %  CHAPTER 5: Relations
    %
    \noindent\section{Relations}
    % 5.1. Definitions and examples of relations
    \wde{5.1.2 Relations between sets} Let $X$ and $Y$ be setss. A \textit{relation between $X$ and $Y$} is a subset of $X \times Y$. When $X = Y$, we call it a \textit{relation on $X$}.
\wde{no number, Opposite relation} Let $X$ and $Y$ be sets and $R \subseteq X \times Y$ a relation. Then $R^{\op}$, the \textit{opposite of $R$}, is a relation from $Y$ to $X$, and $y R^{\op}x \iff xRy$, where $y \in Y$ and $x \in X$.
\wde{5.1.9, Symmetric relation} A relation $R$ on a set $X$ is \textit{symmetric} if $R^{\op} = R$, i.e. if $xRy \iff yRx$ for all $x, y \in X$.
\wl{5.1.11} Let $R$ bea  relation between sets $X$ and $Y$. Then for each $x \in X$, there is a unique subset $R[x]$ of $y$ such that for $y \in Y$, $y \in_Y R[x] \iff xRy$.
\wl{5.1.14} Let $X$, $Y$, and $Z$ be sets, let $R$ be a relation between $X$ and $Y$, and let $S$ be a relation between $Y$ and $Z$. Then there is a unique relation $S \circ R$ between $X$ and $Z$ such that for all $x \in X$ and $z \in Z$, $x(S \circ R)z \iff (xRy \text{ and } ySz \text{ for some } y \in Y)$. We call the relation $S \circ R$ the composite of $R$ and $S$.
\wl{5.1.17} \textbf{(i)} Composite relations are associative. \textbf{(ii)} Let $R$ be a relation between $X$ and $Y$. Then $R \circ \triangle_X = R = \triangle_Y \circ R$. \textbf{(iii)} Let $R$ be a relation between $X$ and $Y$ and $S$ a relation between $Y$ and $Z$. Then $(S \circ R)^{\op} = R^{\op} \circ S^{\op}$.
\wde{5.1.18} A relation $R$ on a set $X$ is: \textbf{(i)} \textit{reflexive} if $xRx$ for all $x \in X$; \textbf{(ii)} \textit{transitive} if $(xRy \text{ and } yRz) \implies xRz$, for $x, y, z \in X$; \textbf{(ii)} \textit{antisymmetric} if $(xRy \text{ and } yRx) \implies x = y$, for $x, y \in X$.
\wde{5.1.21 Order} An \textit{order} (or \textit{order relation} or \textit{ordering}) on a set $X$ is a relation on $X$ that is reflexive, transitive, and antisymmetric. A set together with an order on it is called an \textit{ordered set}.
\wde{5.1.23 Total order} An order $\leq$ on a set $X$ is \textit{total} if for all $x, y \in X$, we have $x \leq y$ or $y \leq x$. Then $X$ together with $\leq$ is called a \textit{totally ordered set}.

    %
    % 5.2 Graphs
    \wde{5.2.1 Graph} Let $f \colon X \to Y$ be a function. The \textit{graph} of $f$ is the function $\Gamma_f = (\id_X, f) \colon X \to X \times Y$. Thus, $\Gamma_f(x) = (x, f(x))$ for all $x \in X$.
\wde{5.2.3 Functional relation} Let $X$ and $Y$ be sets. A relation $R$ between $X$ and $Y$ is \textit{functional} if for all $x \in X$, there exists a unique $y \in Y$ such that $(x, y) \in_{X \times Y} R$.
\wt{5.2.4} Let $X$ and $Y$ be sets. \textbf{(i)} For every function $f \colon X \to Y$, the graph $\Gamma_f$ is a functional relation between $X$ and $Y$. \textbf{(ii)} For every functional relation $R$ between $X$ and $Y$, there is a unique function $f \colon X \to Y$ such thtat the relations $\Gamma_f$ and $R$ are equal.

    %
    % 5.3 Quantifiers
    \wde{no number, $\forall$ and $\exists$} Let $R$ be a relation between sets $X$ and $Y$. Define $\forall_X R = \bigcap_{x \in X} R[x]$, $\exists_X R = \bigcup_{x \in X} R[x]$, which are both subsets of $Y$.
\wl{5.3.1} Let $R$ be a relation between sets $X$ and $Y$. Then for $y \in Y$, \textbf{(i)} $y \in_Y \forall_X R$ if and only if ($xRy$ for all $x \in X$); \textbf{(ii)} $y \in_Y \exists_X R$ if and only if ($xRy$ for some $x \in X$).

    %
    % 5.3 Specifying subsets
    \wde{5.4.2 Property specifying a subset} Let $X$ be a set and let $P(x)$ be a property of elements $x \in X$. We say that $P(x)$ \textit{specifies a subset of $X$} if there exists $A \subseteq X$ such that for $x \in X$, $x \in_X A \iff P(x)$. Then we define $\{x \in X \colon P(x) \}$ to be this subset $A$ of $X$. We say that a subset \textit{exists} if $P(x)$ specifies a subset of $X$.
\wl{5.4.4} For sets and functions $X \rightrightarrows^f_g Y$, the subset $\{x \in X \colon f(x) = g(x) \}$ of $X$ exists. It is called the \textit{equalizer} of $f$ and $g$.
\wl{5.4.7} Let $X$ be a set and let $P(x)$ and $Q(x)$ be properties of elements $x \in X$. If $P(x)$ and $Q(x)$ specify subsets of $X$, then so do the following properties of elements $x \in X$: \textbf{(i)} not P(x); \textbf{(ii)} $P(x)$ and $Q(x)$; \textbf{(iii)} $P(x)$ or $Q(x)$; \textbf{(iv)} $P(x) \implies Q(x)$; \textbf{(v)} $P(x) \iff Q(x)$.
\wl{5.4.10} Let $f \colon X \to Y$ be a function. Let $Q(y)$ be a property of element $y \in Y$, and define a property $P(x)$ of elements $x \in X$ by $P(x) = Q(f(x))$. If $Q(y)$ specifies a subset of $Y$ then $P(x)$ specifies a subset of $X$.
\wl{5.4.12} Let $P(x, y)$ be a property of elements $(x, y) \in X \times Y$ that specifies a subset of $X \times Y$. Then $(\forall x \in X)P(x, y)$ and $(\exists x \in X)P(x, y)$ are properties of elements $y \in Y$ that specify subset of $Y$.

    %
    % 5.3 Specifying funcitons
    \wde{no number, Retraction} For sets $X$ and $Y$, a \textit{retraction} of a function $i \colon Y \to X$ is a function $f \colon X \to Y$ such that $f \circ i = \id_Y$.
\wl{5.5.3.} Every injection with nonempty domain has a retraction.
\wl{5.5.5} Let $X$ be a set. Then there is a function $X \setminus - \colon \Pow(X) \to \Pow(X)$ such that $(X \setminus -) (A) = X \setminus A$ for all $A \in \Pow(X)$.
\wl{5.5.6} Let $X$ be a set. Then there exist functions $\bigcap, \bigcup \colon \Pow(\Pow(X)) \to \Pow(X)$ such that $\bigcap \mathscr{U}$ and $\bigcup \mathscr{U}$ are the subsets of $X$ with the property $x \in_X \bigcap \mathscr{U} \iff (x \in_X A \text{ for all } A \in \mathscr{U})$ (similar for $\bigcup$) for all $\mathscr{U} \in \Pow(\Pow(X))$
\wl{5.5.8} Let $f \colon X \to Y$ be a function. Let $B \xhookrightarrow{j} Y$ be a subset such that $\im f \subseteq_Y B$. Then there is a unique function $f' \colon X \to B$ such taht $j(f'(x)) = f(x)$ for all $x \in X$. The function $f'$ is called the \textit{corestriction} pf $f$ to $B$.
\wpr{5.5.10} Let $f \colon X \to Y$ be a function. Write $j$ for the inclusion $\im f \hookrightarrow Y$. Then there is a unique function $f' \colon X \to \im f$ such thtat the triangle
% https://q.uiver.app/#q=WzAsMyxbMCwwLCJYIl0sWzIsMCwiWSJdLFsxLDEsIlxcdGV4dHtpbX0gZiJdLFswLDEsImYiXSxbMCwyLCJmJyIsMl0sWzIsMSwiaiIsMix7InN0eWxlIjp7InRhaWwiOnsibmFtZSI6Imhvb2siLCJzaWRlIjoidG9wIn19fV1d
\[\begin{tikzcd}[cramped,row sep=tiny]
	X && Y \\
	& {\text{im} f}
	\arrow["f", from=1-1, to=1-3]
	\arrow["{f'}"', from=1-1, to=2-2]
	\arrow["j"', hook, from=2-2, to=1-3]
\end{tikzcd}\]
commutes, and $f'$ is surjective.
\wc{5.5.11} Every function can be expressed as the composite of a surjective followed by an injection.
    %
    %
    %  CHAPTER 6: Gluing
    %
    \noindent\section{Gluing}
    % 6.1 Equivalence relations
    \wde{6.1.1 Equivalence relation} An \textit{equivalence relation} on a set $X$ is a reflexive, transitive, symmetric relation on $X$.
\wl{6.1.4} The intersection of any set of equivalence relations is an equivalence relation. That is, let $X$ be a set, let $\mathscr{E} \subseteq \Pow(X \times X)$, and suppose that each element $E \in \mathscr{E}$ is an equivalence relation on X. Then $\bigcap\mathscr{E}$ is an equivalence relation on $X$.
\wde{6.1.5 Generated equivalence relation} Let $R$ be a relation on a set $X$. The \textit{equivalence relation generated by $R$} is the intersection of all equivalence relations on $X$ containing $R$. That is, the equivalence relation generated by $R$ is $\bigcap\{ E \in \Pow(X \times X) \colon E \text{ is an equiv. rel. on } X \text{ and } R \subseteq E \}$.
\wl{6.1.6} Let $R$ be a relation on a set $X$, and let $\sim$ be the equivalence relation on $X$ generated by $R$. Then: \textbf{(i)} $\sim$ is an equivalence relation containing $R$; \textbf{(ii)} any equivalence relation containing $R$ also contains $\sim$. I.e. $\sim$ is the smallest equivalence relation containing $R$.

    %
    % 6.2 Quotients
    \wde{n.n., Equivalence class} Let $E$ be an equiv. rel. on a set $X$. The \textit{equiv. class} of an elem. $x \in X$ is the subset $E[x]$ of $X$, as defined in lemma 5.1.11. If we write the equiv. rel. as $\sim$, then we denote the equiv. class of $x$ by $[x]_{\sim}$ or just $[x]$.
\wl{6.2.3} Let $\sim$ be an equiv. rel. on a set $X$. Then for $x, y, \in X$, $x \sim y \iff [x] = [y]$.
\wde{n.n., Quotient} Let $E$ be an equiv. rel. on a set $X$. Then $E$ corresponds to a function $\bar{\epsilon} X \to \Pow(x)$, $x \mapsto E[x]$. Define $X/E$ the \textit{quotient of $X$ by $E$} to be the image of $\bar{\epsilon}$.
\wl{n.n.} There is a unique function $\pi_E \colon X \to X/E$ such that $\pi_E(x) = E[X]$ for all $x \in X$; that is, the diagram [right in prop 5.5.10] commutes. We call $\pi_E$ the \textit{natural surjection}, \textit{canonical surjection}, or \textit{quotient map}.
\wlog{6.2.5} Let $\sim$ be an equiv. rel. on a set $X$. Then for $x, y \in X$, $\pi_{\sim}(x) = \pi_{\sim}(y) \iff x \sim y$.
\wpr{6.2.8, Universal property of quotients} Let $\sim$ be an equiv. rel. on a set $X$. Let $Y$ be a set and $f \colon X \to Y$ a function such that for $x, x' \in X$, $x \sim x' \implies f(x) = f(x')$. Then there is a unique function $\bar{f} \colon X/\sim \to Y$ such that $\bar{f}(\pi_{\sim}(x)) = f(x)$ for all $x \in X$. As a commutative diagram (left):
% https://q.uiver.app/#q=WzAsMyxbMCwwLCJYIl0sWzAsMSwiWC9cXHNpbSJdLFsxLDEsIlkiXSxbMCwxLCJcXHBpX3tcXHNpbX0iLDJdLFsxLDIsIlxcYmFye2Z9IiwyLHsic3R5bGUiOnsiYm9keSI6eyJuYW1lIjoiZG90dGVkIn19fV0sWzAsMiwiZiJdXQ==
% https://q.uiver.app/#q=WzAsMyxbMCwwLCJYIl0sWzAsMSwiWC9cXHNpbV9mIl0sWzEsMSwiWSJdLFswLDEsIlxccGlfe1xcc2ltX2Z9IiwyXSxbMSwyLCJcXGJhcntmfSIsMix7InN0eWxlIjp7ImJvZHkiOnsibmFtZSI6ImRvdHRlZCJ9fX1dLFswLDIsImYiXV0=
\[\begin{tikzcd}[cramped]
    X \\
    {X/\sim} & Y
    \arrow["{\pi_{\sim}}"', from=1-1, to=2-1]
    \arrow["f", from=1-1, to=2-2]
    \arrow["{\bar{f}}"', dotted, from=2-1, to=2-2]
\end{tikzcd}
\begin{tikzcd}[cramped]
    X \\
    {X/\sim_f} & Y
    \arrow["{\pi_{\sim_f}}"', from=1-1, to=2-1]
    \arrow["f", from=1-1, to=2-2]
    \arrow["{\bar{f}}"', "\cong", dotted, from=2-1, to=2-2]
\end{tikzcd}\]
\wt{6.2.11, First isomorphism theorem} Let $X$ and $Y$ be sets, and let $f \colon X \to Y$ be a surjection. Then there is a unique isomorphism $\bar{f} \colon X/\sim_f \to Y$ such that $\bar{f}(\pi_{\sim_f}(x)) = f(x)$ for all $x \in X$. In particular, $X/\sim_f \cong Y$. As a commutative diagram: [above right]

    %
    % 6.3 Disjoint unions and coproducts
    \wde{6.3.1 Disjoint union diagram} A \textit{disjoint union diagram} consists of sets and functions $X \xrightarrow{i} D \xleftarrow{j} Y$ with the following properties: \textbf{(i)} $i$ and $j$ are injective; \textbf{(ii)} the union of $X \xrightarrow{i} D$ and $Y \xrightarrow{j} D$ (seen as subsets of $D$) is $D$; \textbf{(iii)} the intersection of $X \xrightarrow{i} D$ and $Y \xrightarrow{j} D$ is empty (they are \textit{disjoint}).

    %
    % 6.4 Comparing sizes of sets
    \wde{6.4.1 Set $\leq$} For sets $X$ and $Y$, write $X \leq Y$ if there exists an injection $X \to Y$.
\wl{6.4.2} Let $X$, $Y$, and $Z$ be sets. \textbf{(i)} If $X \leq Y$ and $Y \leq Z$ then $X \leq Z$. \textbf{(ii)}If $X \cong Y$ then $X \leq Y$.
\wl{6.4.3} Let $X$ and $Y$ be sets. Then $X \leq Y$ iff there exists a set $A$ such that $X + A \cong Y$.
\wde{n.n., Set $<, \geq, >$} We write $X < Y$ is $X \leq Y$ but $X \not \cong Y$, and we define $\geq$ and $>$ as you would imagine.
\wt{6.4.4 Cantor} $X < \Pow(X)$ for all sets $X$.
\wde{6.4.6 Order-preserving function} Let $P$ be a set with an order $\leq$, and let $Q$ be a set with an order $\trianglelefteq$. A function $\varphi \colon P \to Q$ is \textit{order-preserving} if $p \leq p' \implies \varphi(p) \trianglelefteq \varphi(p')$ for all $p, p' \in P$.
\wpr{6.4.8 Knaster-Tarski fixed point} Let $X$ be a set, and let $\varphi \colon \Pow(X) \to \Pow(X)$ be a function that is order-preserving with respect to $\subseteq_X$. Then there exists $A \subseteq X$ such that $\varphi(A) = A$.
\wt{6.4.10 Cantor-Bernstein} Let $X$ and $Y$ be sets. If $X \leq Y$ and $Y \leq X$ then $X \cong Y$.

    %
    % 6.5 Families
    \wde{6.5.4} Let $I$ be a set. An \textit{$I$-indexed family of sets} is a set $X$ with a function $p \colon X \to I$. We call $I$ the \textit{indexing set}, the fibre $p^{-1}(i)$ as $X_i$, and call $X_i$ the $i$th \textit{member} of the family.
\wl{6.5.6} Let $Y$ and $I$ be sets, and let $(A_i \hookrightarrow Y)_{i \in I}$ be an $I$-indexed family of subsets of $Y$. Then $\exists$ an $I$-indexed family of sets $X \to I$ such that $X_i \cong A_i$ for all $i \in I$.
\wde{6.5.8 Family isomorphism} Let $I$ be a set. Two $I$-indexed families $X \xrightarrow{p} I$ and $X' \xrightarrow{p'} I$ are \textit{isomorphic} if they are iso. over $I$ (definition 3.2.12). We often write $X \xrightarrow{p} I$ as $(X_i)_{i \in I}$ and call $p$ the \textit{projection} of the family.
\wpr{6.5.12} Let $(X_i)_{i \in I}$ be a family of sets. Then $\exists$ a set not iso. to any of the sets $X_i (i \in I)$.

    %
    %
    %  CHAPTER 7: Number systems
    %
    \noindent\section{Number systems}
    % 7.1 The natural numbers
    % This is the complete chapter.
    % \wt{7.1.1 Principle of induction} Let $A \subseteq \N$. Suppose that $0 \in_{\N} A$ and that for all $n \in \N$, we have $n \in_{\N} A \implies s(n) \in_{\N} A$. Then $A = \N$.
\wl{7.1.3} The function $\binom{0}{s} \colon \mathbf{1} + \N \to \N$ is an isomorphism.
\wpr{7.1.4} \textbf{(i)} Every natural number is $0$ or a successor. \textbf{(ii)} $0$ is not a successor. \textbf{(iii)} If $s(n) = s(m)$ then $n = m$, for $n, m \in N$.
\wl{7.1.5} There exists unique function $+ \colon \N \times \N \to \N$ such that $m + 0 = m$ and $m + s(n) = s (m + n)$ for all $m, n \in \N$.
\wl{7.1.6} $\exists!$ function $\cdot \colon \N \times \N \to \N$ such that $m \cdot 0 = 0$ and $m \cdot s(n) = m \cdot n + m$ for all $m, n \in \N$.
\wde{n.n.} We call $+$ \textit{addition} and $\cdot$ \textit{multiplication} on $\N$ and we define $1$ as $s(0)$.
\wl{7.1.7} $\exists!$ function $E \colon \N \times \N \to \N$ such that, writing $E(m, n)$ as $m^n$, $m^0 = 1$ and $m^{s(n)} = m^n \cdot m$ for all $m, n \in \N$.
\wl{7.1.8} $s(n) = n + 1$ for all $n \in \N$.
\wde{7.1.9 $\leq$ on $\N$} The relation $\leq$ on $\N$ is defined by $m \leq n \iff k + m = n$ for some $k \in N$. We define $<$, $\geq$, and $>$ as usual.
\wl{7.1.10} Let $m, n \in \N$. \textbf{(i)} $m \leq n \iff s(m) \leq s(n)$. \textbf{(ii)} $m < n \iff s(m) < s(n)$. \textbf{(iii)} $n \geq 0$ \textbf{(iv)} $n \leq 0 \implies n = 0$ \textbf{(v)} $m < s(n) \iff m \leq n$.
\wde{7.1.12} For $n \in \N$, we write $B(n) = \{ i \in \N \colon i < n \}$.
\wl{7.1.13} $B(s(n)) \cong B(n) + \mathbf{1}$ for all $n \in \N$.
\wpr{7.1.14} For all $m, n \in \N$, \textbf{(i)} $B(m + n) \cong B(m) + B(n)$, and $B(0) \cong \emptyset$; \textbf{(ii)} $B(m \cdot n) \cong B(m) \times B(n)$, and $B(1) \cong \mathbf{1}$; \textbf{(iii)} $B(m^n) \cong B(m)^{B(n)}$.
\wl{7.1.15} Let $X$ and $Y$ be sets. Then: \textbf{(i)} $X \cong Y \iff X + \mathbf{1} \cong Y + \mathbf{1}$; \textbf{(ii)} $X \leq Y \iff X + \mathbf{1} \leq Y + \mathbf{1}$.
\wl{7.1.16} $m \leq n \iff B(m) \leq B(n)$, for $m, n \in \N$.
\wl{7.1.17} $m = n \iff B(m) \cong B(n)$, for $m, n \in \N$.
\wpr{7.1.18} On $\N$, addition and multiplication are commutative, associative, have identity ($0$ and $1$ respectively). Multiplication distributes over addition. We have standard power laws too. Importantly, $p^0 = 1$ for all $p \in \N$.
\wpr{7.1.19} \textbf{(i)} The relation $\leq$ on $\N$ is an order. \textbf{(ii)} Let $m, n \in \N$ with $m \leq n$. Then for all $p \in \N$ we have $m + p \leq n + p$, $mp \leq np$, and $m^p \leq n^p$; if also $p > 0$ then $p^m \leq p^n$.
\wl{7.1.21} Let $m, n \in \N$. If $m + n = 0$ then $m = n = 0$, and if $mn = 0$ then $m = 0$ or $n = 0$.
\wl{7.1.22} Let $m, n, p \in \N$. If $m + p = n + p$ then $m = n$.
\wl{7.1.24} Let $m, n, p \in \N$ with $p \neq 0$. If $m p = n p$ then $m = n$.
\wl{7.1.26} The relation $\leq$ on $\N$ is total.

    % This is a minimal one with only important results.
    \wt{7.1.1 Principle of induction} Let $A \subseteq \N$. Suppose that $0 \in_{\N} A$ and that for all $n \in \N$, we have $n \in_{\N} A \implies s(n) \in_{\N} A$. Then $A = \N$.
\wl{7.1.3} The function $\binom{0}{s} \colon \mathbf{1} + \N \to \N$ is an iso.
\wde{7.1.12} For $n \in \N$, we write $B(n) = \{ i \in \N \colon i < n \}$.
\wl{7.1.13} $B(s(n)) \cong B(n) + \mathbf{1}$ for all $n \in \N$.
\wpr{7.1.14} For all $m, n \in \N$, \textbf{(i)} $B(m + n) \cong B(m) + B(n)$, and $B(0) \cong \emptyset$; \textbf{(ii)} $B(m \cdot n) \cong B(m) \times B(n)$, and $B(1) \cong \mathbf{1}$; \textbf{(iii)} $B(m^n) \cong B(m)^{B(n)}$.
\wl{7.1.15} Let $X$ and $Y$ be sets. Then: \textbf{(i)} $X \cong Y \iff X + \mathbf{1} \cong Y + \mathbf{1}$; \textbf{(ii)} $X \leq Y \iff X + \mathbf{1} \leq Y + \mathbf{1}$.
\wl{7.1.16, 7.1.17} $m \leq n \iff B(m) \leq B(n)$ and $m = n \iff B(m) \cong B(n)$, for $m, n \in \N$.
\wl{7.1.26} The relation $\leq$ on $\N$ is total.

    %
    % 7.2 Induction and recursion.
    \wde{7.2.2 $\ddownarrow$} Let $\leq$ be an order relation on a sset $X$. For $x \in X$, we write $\ddownarrow x$ for the subset $\{ x \in X \colon y < x \}$ of $X$.
\wt{7.2.4 Principle of strong induction} Let $A$ be a subset of $\N$ such that for all $n \in \N$, $(i \in A \text{ for all natural numbers } i < n) \implies n \in AA$. Then $A = \N$.
\wpr{7.2.6} Let $X$ be a set and $(r_n)_{n \in \N} \in \prod_{n \in \N} X^{(X^{\mathbf{n}})}$. Then there exists a unique function $f \colon \N \to X$ such that $f(n) = r_n (f \mid_{\ddownarrow_n})$ for all $n \in \N$.

    %
    % 7.3 The integers
    \wde{no number} For section 2.3, we define $\sim$ for the relation on $\N \times \N$ defined by $(m, n) \sim (m', n') \iff m + n' = m' + n$ ($m, m', n, n' \in \N$).
\wl{7.3.1} The relation $\sim$ on $\N$ is an equivalence relation.
\wde{7.3.2 $\Z$} The set $\Z$ of \textit{integers} is $(\N \times \N) / \sim$.
\wl{7.3.5} Let $\sim_1$ be an equivalence relation on a set $X$ and $\sim_2$ an equivalence relation on a set $Y$. Then the relation $\sim$ on $X \times Y$ defined by $(x, y) \cong (x', y') \iff x \sim_1 x' \text{ and } y \sim_2 y'$ ($x, x' \in X, y, y' \in Y$) is an equivalence relation. Moreover, $(X \times Y) / \sim \cong X / \sim_1 \times Y / \cong_2$.
\wpr{7.3.9} $\Z$ is a commutative ring.
\wde{7.3.11 $\leq$ on $\Z$} $a \leq b \iff b = n + a$ for some $n \in \N$.
\wl{7.3.12} $\leq$ is a total order on $\Z$.
\wl{7.3.13} $\Z$ with $\leq$ is an ordered ring.

    %
    % 7.3 The rational numbers
    \wde{n.n.} We define $\Z^* = \Z \setminus \{ 0 \}$. For section 7.4, $\sim$ is the relation on $\Z \times \Z^*$ defined by $(a, b) \sim (a', b') \iff ab' = a'b$ ($a, a' \in \Z, b, b' \in \Z^*$).
\wl{7.4.1} $\sim$ is an equivalence relation on $\Z \times \Z^*$.
\wde{7.4.3 $\Q$} The set $\Q$ of \textit{rational numbers} is $(\Z \times \Z^*) / \sim$.
\wpr{7.4.9} $\Q$ is a field.
\wl{7.4.12} $\leq_{\Q}$ is a total order on $\Q$.
\wl{7.4.13} $\Q$ is an ordered field.

    %
    % 7.3 The real numbers
    \wde{7.5.1 Lower \& upper bound} Let $X$ be a set with an order $leq$, and let $A \subseteq X$. A \textit{lower bound} of $A$ in $X$ is an element $\ell \in X$ such that $\ell \leq a$ for all $a \in A$. We write $\lb(A) = \{ \ell \in X \colon \ell \text{ is a lower bound of } A \}$. The set $\ub(A)$ of \textit{upper bounds} is defined similarly.
\wde{7.5.3 Dedekind cut} A \textit{Dedekind cut} is a pair $(L, U) \in \Pow(\Q) \times \Pow(\Q)$ such that $L = \lb(U)$ and $U = \ub(L)$. It is \textit{proper} if $L$ and $U$ are nonempty.
\wde{7.5.7 $\R$} The set $\R$ of \textit{real numbers} is $\{ (L, U) \in \Pow(\Q) \times \Pow(\Q) \colon (L, U) \text{ is a proper Dedekind cut } \}$. The inclusion $i \colon \Q \to \R$ is defined by $i(q) = (\downarrow q, \uparrow q)$ where $\downarrow q = \{ r \in \Q \colon r \leq q \}$. It is injective, by antisymmetry of $\leq_{\Q}$.
\wl{7.5.8} Let $(L, U)$ be a Dedekind cut. Then the subset $L$ of $\Q$ is \textit{downwards closed}; that is, if $q, r \in \Q$ with $q \leq r$ and $r \in L$, then $q \in L$. Similarly, $U$ is upwards closed.
\wde{7.5.9 $\leq$ on $\R$} The relation $\leq_R$ on $\R$ is defined by $(L, U) \leq_{\R} (L', U') \iff L \subseteq L'$ for $(L, U), (L', U') \in \R$.
\wl{7.5.11} The relation $\leq_{\R}$ on $\R$ is a total order.
\wde{no number, least upper bound} Let $A \subseteq \R$. A \textit{least upper bound} of $A$ in $\R$ is a least element of the set of upper bounds of $A$ in $\R$, where an element $a \in A$ is \textit{least} if $a \leq b$ for all $b \in A$.
\wt{7.5.13} Every nonempty subset of $\R$ that has an upper bound has a least upper bound. We say that $\R$ is a (Dedekind) complete ordered field.
\wc{no number} Up to isomorphism, there is only one complete ordered field: $\R$.

    %
    %
    %  CHAPTER 8: Well ordered sets
    %
    \noindent\section{Well ordered sets}
    % 8.1 Definitions and examples
    \wde{no number, Order isomorphism} An \textit{isomorphism of ordered sets} or \textit{order isomorphism} $X \to Y$ is an order-preserving bijection whose inverse is also order-preserving. We say that $X$ and $Y$ are \textit{order isomorphic}, or just \textit{isomorphic} and write $X \cong Y$.
\wde{8.1.3 Well order} An order $\leq$ on a set $W$ is a \textit{well order} if every nonempty subset of $W$ has a leaast element. A set together with a well order is called a \textit{well ordered set}.
\wl{8.1.4} Every well ordered set is totally ordered.
\wde{8.1.7 Ordinal sum and product} Let $X = (X, \leq_1)$ and $Y = (Y, \leq_2)$ be ordered sets. \textbf{(i)} The \textit{ordinal sum} of $X$ and $Y$ is the coproduct $X + Y$ with the following order relation $\leq$: for $z, z' \in X + Y$, we define $z \leq z'$ if and only if $(z, z' \in X \text{ and } z \leq_1 z')$ or $(z, z' \in Y \text{ and } z \leq_2 z')$ or $(z \in X \text{ and } z' \in Y)$. \textbf{(i)} The \textit{ordinal product} of $X$ and $Y$ is the product $X \times Y$ with the following order relation $\leq$: for $(x, y), (x', y') \in X \times Y$, we define $(x, y) \leq (x', y')$ if and only if $(y <_2 y')$ or $(y = y' \text{ and } x \leq_1 x')$. We call the ordinal product ordering the \textit{reverse lexicographic order}.
\wl{8.1.8} The ordinal sum and ordinal product of two well ordered sets are well ordered.
\wde{no number, Successor and limit} A well ordered set is called a \textit{successor} if it is isomorphic to $V + \mathbf{1}$ for some well ordered set $V$, and a \textit{limit} otherwise.
\wl{8.1.13} A well ordered set is a successor if and only if it has a greatest element.
\wde{8.1.15 Downset} A subset $A$ of an ordered set $X$ is \textit{downwards closed}, or a \textit{downset}, if for all $x, y \in X$, $(x \leq y \text{ and } y \in A) \implies x \in A$.
\wl{8.1.17} Let $W$ be a well ordered set and let $A$ be a downset in $W$. Then $A = W$ or there exists a unique $w \in W$ such that $A = \ddownarrow w$.
\wt{8.1.18 Transfinite induction} Let $W$ be a well ordered set. Let $A$ be a subset of $W$ such that for all $w \in W$, $\ddownarrow w \subseteq A \implies w \in A$. Then $A = W$.

    %
    % 8.2 Comparing well ordered sets
    \wde{8.2.1} Let $V$ and $W$ be w.o. sets. A \textit{map of w.o. sets} (or just a \textit{map}) from $V$ to $W$ is an order-preserving injection whose image is downwards closed.
\wl{8.2.4} Let $f, g \colon V \to W$ be maps of w.o. sets. Then $f = g$.
\wde{8.2.5 $\preceq$} For w.o. sets $V$ and $W$, we write $V \preceq W$ if there exists a map of w.o. sets $V \to W$.
\wl{8.2.7} Let $V$ and $W$ be w.o. sets. Then $V \preceq W$ iff $V$ is isomorphic to a proper downset of $W$.
\wc{8.2.8} Let $V$ and $W$ be w.o. sets. If $V$ is isomorphic to a proper downset of $W$ then $V \not \cong W$.
\wl{8.2.11} \textbf{(i)} $V \cong W \implies V \preceq W$; \textbf{(ii)} $(V \preceq W \text{ and } W \preceq X) \implies V \preceq X$; \textbf{(iii)} $(V \preceq W \text{ and } W \preceq V) \implies V \cong W$. 
\wl{8.2.12} For w.o. sets $V, W$, either $V \preceq W$ or $W \preceq V$.
\wt{8.2.13} Let $I$ be a nonempty set and $(W_i)_{i \in I}$ a family of w.o. sets. Then there is some $i \in I$ s.t. $\forall j \in I$, we have $W_i \preceq W_j$.
\wpr{8.2.16} Let $W$ be a w.o. set. Let $A$ be a subset of $W$ with the induced order. Then $A \preceq W$.

    %
    % 8.3 The Hartogs theorem
    \wt{8.3.1 Hartogs} For every set $X$, there exists a well ordered set $(H, \leq)$ such that $H \not \leq X$.

    %
    % 8.4 Chains and their upper bounds
    \wde{8.4.1 Chain} Let $(X, \leq)$ be an ordered set. A \textit{chain} in $X$ is a subset $C \subseteq X$ such that for all $x, y \in C$, either $x \leq y$ or $y \leq x$.
\wpr{8.4.3} Let $(X, \leq)$ be an ordered set and let $\varphi \colon \{ \text{chains in } X \} \to X$ be a function assigning to each chain $C$ an upper bound $\varphi(C) \in X$. Then $\varphi(C) \in C$ for some chain $C$.

    %
    %
    %  CHAPTER 9: The axiom of choice
    %
    \noindent\section{The axiom of choice}
    % 9.1 Easy equivalents of the axiom of choice
    \wde{9.1.11 AoC equivalents} A statement $S$ is \textit{equivalent to the AoC} if \textbf{(i)} Axioms 1-9 together with the AoC imply $S$, and \textbf{(ii)} Axioms 1-9 together with $S$ imply the AoC.
\wde{n.n.} For a family $X \xrightarrow{p} I$, we define its \textit{product} to be $\prod_{i \in I} X_i = \{ \text{sections of p} \} = \{ x \in X^I \colon p \circ x = \id_I \}$. We write elements $x \in \prod_{i \in I} X_i$ as $(x_i)_{i \in I}$, where $x_i = x(i)$.
\wde{n.n., Total relation} A relation $R$ between sets $X$ and $Y$ is \textit{total} if $\forall x \in X$, there exists $y \in Y$ such that $xRy$.
\wde{n.n.} $f \colon X \to Y$ \textit{refines} $R$ if $xRf(x)$ $\forall x \in X$.
\wde{n.n.} A \textit{choice function} on $X$ is a function $f \colon \Pow(X) \setminus \emptyset \to X$ such that $f(A) \in A$ $\forall A \in \Pow(X) \setminus \emptyset$.
\wde{n.n.} A \textit{system of representatives} for $\sim$ is a subset $A$ of $X$ with the following property: $\forall x \in X$, $\exists! a \in A$ such that $x \sim a$.
\wpr{9.1.4} These are equivalent to the AoC. \textbf{(i)} The product of any family of nonempty sets is nonempty. \textbf{(ii)} Every total relation can be refined to a function. \textbf{(iii)} Every set has a choice function. \textbf{(iv)} Every equiv. rel. has a sys. of repr.

    %
    % 9.2 Zorn's lemma
    \wde{no number, Maximal} An element $m$ of an ordered set $X$ is \textit{maximal} if there is no element $x \in X$ such that $m < x$.
\wde{no number, Greatest} A \textit{greatest} element of an ordered set $X$ is an element $g$ such that $x \leq g$ for all $x \in X$.
\wt{9.2.4 Zorn's lemma} Let $X$ be an ordered set in which every chain has an upper bound. Then $X$ contains a maximal element.
\wc{9.2.9} Let $X$ be an ordered set in which every chain has an upper bound. Let $x \in X$. Then $X$ contains a maximal element $m$ such that $m \geq x$.
\wt{9.2.11 Cardinal comparability} For all sets $X$ and $Y$, either $X \leq Y$ or $Y \leq X$.

    %
    % 9.3 Harder equivalents of the axiom of choice
    \wt{9.3.1} Each of the following statements is equivalent to teh axiom of choice. \textbf{(i)} Zorn's lemma; \textbf{(ii)} cardinal comaprability; \textbf{(iii)} the \textit{well ordering principle}: for every set $X$ there exists a well ordering on $X$.
\wt{9.3.4} Let $(X_i)_{i \in I}$ be a family of sets, with $I$ nonempty. Then there is some $i \in I$ such that for all $j \in I$, we have $X_i \leq X_j$.

    %
    % 9.4 Unnecessary uses of the axiom of choice
    \wde{9.4.2 Finite} A set $I$ is \textit{finite} if there exists $n \in \N$ such that $I \cong \mathbf{n}$.
\wl{9.4.3} Let $p \colon X \to I$ be a surjection form a set $X$ to a finite set $I$. Then axioms 1-9 imply that $p$ has a section.

    %
    %
    %  CHAPTER 10: Cardinal arithmetic
    %
    \noindent\section{Cardinal arithmetic}
    % 10.1 Finite and infinite
    \wc{Recall:} $\mathbf{m} \leq \mathbf{n} \iff m \leq n$; $\mathbf{m} \cong \mathbf{n} \iff m = n$.
\wde{n.n., Infinite} A set is infinite if it is not finite.
\wl{Remark 10.1.1 Pigeonhole principle} Let $m, n \in \N$. If $m > n$ then there is no injection $\mathbf{m} \to \mathbf{n}$.
\wde{n.n., Cardinality} The \textit{cardinality} of a finite set $X$, writen as $|X|$, is the unique natural number $n$ such that $X \cong \mathbf{n}$.
\wl{10.1.2} Let $X$ and $Y$ be sets, with $X$ finite. If there exists an injection $Y \to X$ or a surjection $X \to Y$, then $Y$ is also finite.
\wl{10.1.4} Let $X$ and $Y$ be sets. Then $X \leq Y$ iff $X$ is empty or there exists a surjection $Y \to X$.
\wpr{10.1.5} Let $X$ be a set. The following conditions are equivalent: \textbf{(i)} $X$ is infinite; \textbf{(ii)} $\N \leq X$, that is, there exists an injection $\N \to X$; \textbf{(iii)} there exists a surjection $X \to \N$; \textbf{(iv)} $X \cong X + \mathbf{1}$; \textbf{(v)} there exissts a non-surjective injection $X \to X$; \textbf{(vi)} there exists a non-injective surjection $X \to X$.
\wc{10.1.8} Every subset of $\N$ is either isomorphic to $\N$ or finite.

    %
    % 10.2 Countable and uncountable
    \wde{10.2.1, (Un)countable} A set $X$ is \textit{countable} if $X \leq \N$, and \textit{uncountable} otherwise.
\wl{10.2.3} A set $X$ is countable if and only if $X$ is empty or there exists a surjection $\N \to X$.
\wc{10.2.4} Let $X$ and $Y$ be sets, with $X$ countable. If there exists an injection $Y \to X$ or a surjection $X \to Y$, then $Y$ is also countable.
\wpr{10.2.5} $\N \times \N \cong \N$.
\wl{10.2.7} \textbf{(i)} For all ssets $X, X', Y, Y'$, if $X \leq X'$ and $Y \leq Y'$ then $X + Y \leq X' + Y'$, and $X \times Y \leq X' \times Y'$. \textbf{(ii)} For all ssets $X, X', Y$, if $X \leq X'$ then $X^Y \leq X'^Y$, and $Y^X \leq Y^{X'}$, the latter under the restriction that $Y$ is nonempty.
\wc{10.2.8} Let $n \geq 1$. Let $X_1, \dots, X_n$ be countable sets. Then $X_1 \times \dots \times X_n$ is countabler.
\wde{no number} Given a set $X$, we define $X^n$ for each $n \in \N$ recursively as follows: $X^0 = \mathbf{1}$, and $X^{n + 1} = X^n \times X$ ($n \in \N$). Thus, $X^1 \cong X$, $X^2 \cong X \times X$, and so on.
\wc{10.2.9} Let $X$ be a countable set. Then $X^n$ is countable for all $n \geq 1$.
\wl{10.2.11} Let $X$ be a set, and let $(A_i)_{i \in I}$ be a family of subsets of $X$. If $I$ is countable and $A_i$ is countable for each $i \in I$, then so too is $\bigcup_{i \in I} A_i$.
\wl{10.2.14} $\R$ is isomorphic to any real interval with more than one element.
\wpr{10.2.15} $R \cong \mathbf{2}^{\N}$. In particular, $\R$ is uncountable.
\wc{10.2.17} A real interval with more than one element is uncountable.

    %
    % 10.3 Sums and products
    \wpr{10.3.4} $X \times X \cong X$ for every infinite set $X$.
\wc{10.3.5} $X^n \cong X$ for every infinite set $X$ and integer $n \geq 1$.
\wde{no number, $\max$} For sets $X$ and $Y$, we define $max(X, Y) = X$ if $X \geq Y$, $max(X, Y) = Y$ if $Y \geq X$.
\wt{10.3.9} Let $X$ and $Y$ be nonempty sets, at least one of which is infinite. Then $X \times Y \cong X + Y \cong \max(X, Y)$.

    %
    % 10.3 Powers and beyond.
    \wl{10.4.1} Let $X$ be a set. Then $X^{\mathbf{n}} \cong X^n$ for all $n \in \N$.
\wpr{10.4.2} $X^A \cong X$ for all infinite sets $X$ and finite nonempty sets $A$.
\we{10.4.3} $\R^{\N} \cong \mathbf{2}^{\N} \cong \R$.
\wl{10.4.5} Let $X$ and $Y$ be sets, with $X$ infinite and $X \geq Y \geq \mathbf{2}$. Then $Y^X \cong \mathbf{2}^X$. In particular, $X^X \cong \mathbf{2}^X$ for all infinite sets $X$.
\we{10.4.6} \textbf{(i)} $\N^{\N} \cong \mathbf{2}^{\N} \cong \R$. \textbf{(ii)} $\R^{\R} \cong \mathbf{2}^{\R} \cong \mathbf{2}^{\mathbf{2}^{\N}}$.

  \end{multicols}
\end{document}
